%!TEX root = ../main.tex
\chapter{Viabilidade e Pertinência}

\hl{Neste segundo capítulo deverá ser demonstrada viabilidade e relevância do projeto. A viabilidade deverá ser avaliada por critérios econométricos, demonstrando-se que a solução proposta terá características para poder ser continuada após conclusão do TFC, não se esgotando enquanto projeto académico. 
Na componente de pertinência e relevância, os alunos deverão demonstrar que o trabalho em desenvolvimento tem impacto positivo e contribui para a resolução do problema identificado no capítulo anterior. A demonstração deve apresentar dados concretos e verificáveis, preferencialmente de fontes externas ao TFC (e.g.: estudos de mercado; questionários a stakeholders ou utilizadores potenciais; opinião de especialistas reconhecidos; etc.) 
Valorizam-se trabalhos que apresentem validação por terceiros. Nestes casos, deverá ser realizado questionário de viabilidade, interesse e pertinência, aplicado à população alvo identificada e analisados os resultados obtidos. Questionário, incluindo fundamentação, e análise devem ser apresentados no anexo referente ao estudo de viabilidade}