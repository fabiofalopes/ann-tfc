\chapter{Introdução}
\label{cha:introducao}

Considerando os atuais desafios da área de Processamento de Linguagem Natural (PLN) e acompanhando as tendências emergentes, surgiu, no contexto do AISIC Lab (Artificial Intelligence, Social Interaction and Complexity), a necessidade de uma ferramenta especializada para a anotação de dados. Este projeto visa responder aos desafios específicos encontrados no laboratório, nomeadamente na análise de interações em ambientes de chat com múltiplos participantes, uma tarefa conhecida como \textit{chat disentanglement}.

\section{Enquadramento e Motivação}

A anotação de dados é um processo crítico no treino de modelos de \textit{Machine Learning}, sendo indispensável para transformar dados não estruturados — como o texto de uma conversa — em dados estruturados e anotados. São estes dados que servem como "verdade fundamental" (\textit{ground truth}), permitindo que os algoritmos de PLN aprendam a reconhecer padrões e a executar tarefas complexas com precisão.

Uma das tarefas mais desafiadoras neste domínio é o \textbf{\textit{chat disentanglement}}: o processo de separar um diálogo, que pode conter múltiplas conversas entrelaçadas, nos seus \textit{threads} (fios de conversa) constituintes. A motivação principal deste projeto emerge da ausência de ferramentas que sejam, simultaneamente, especializadas nesta tarefa e que integrem mecanismos de controlo de qualidade, como a análise da concordância entre anotadores (\textit{Inter-Annotator Agreement} - IAA).

Uma das motivações deste projeto emerge das necessidades específicas identificadas no AISIC LAB, onde a análise de mensagens em grupos de chat, requer ferramentas especializadas de anotação. 
Através de análises preliminares e feedback dos investigadores, identificámos três desafios fundamentais:

\begin{enumerate}
    \item \textbf{Complexidade das Tarefas de Anotação}:
    \begin{itemize}
        \item Necessidade de suporte a diferentes tipos de anotação
        \item Gestão de múltiplos anotadores e controle de qualidade
        \item Requisitos específicos para diferentes domínios de aplicação
    \end{itemize}

    \item \textbf{Limitações das Ferramentas Existentes}:
    \begin{itemize}
        \item Ferramentas estabelecidas como o BRAT\footnote{\cite{brat-repo,brat-about}} apresentam limitações tecnológicas e falta de evolução
        \item Soluções atuais como o Doccano\footnote{\cite{doccano-repo,doccano-docs}}, que utiliza Django como framework backend, impõem uma estrutura mais rígida, o que limita a capacidade de realizar adaptações específicas às necessidades do projeto.
        \item Necessidade de maior flexibilidade na modelação de dados e lógica aplicacional
        \item Dificuldade de integração com fluxos de trabalho existentes e específicos
    \end{itemize}

    \item \textbf{Necessidade de Integração Completa do Workflow}:
    \begin{itemize}
        \item Gestão integrada das fases pré-anotação, anotação e pós-anotação
        \item Necessidade de implementação de métricas e análises específicas
        \item Suporte a fluxos customizados de processamento de dados
        \item Integração com pipelines de NLP e análise de dados existentes
    \end{itemize}
\end{enumerate}

A decisão de desenvolver uma solução dedicada, em vez de adaptar ferramentas existentes, fundamenta-se na necessidade de maior controlo sobre todo o processo de anotação. Esta abordagem garante a flexibilidade para implementar workflows customizados de pré-processamento, anotação e análise posterior dos dados. 

O problema central que este trabalho se propõe a resolver é, portanto, o seguinte: **como criar um ambiente de software integrado que não apenas facilite a tarefa de anotação de \textit{chat disentanglement}, mas que também forneça aos gestores de projeto as ferramentas para gerir o processo e avaliar a qualidade das anotações produzidas?**


\section{Objetivos}

Para responder ao problema identificado, foram definidos e alcançados os seguintes objetivos específicos para o projeto:

\begin{enumerate}
    \item \textbf{Desenvolver uma Ferramenta Dedicada:} Construir uma aplicação web completa, com um \textit{frontend} reativo e um \textit{backend} robusto, especificamente desenhada para a tarefa de \textit{chat disentanglement}.
    
    \item \textbf{Implementar Funcionalidades de Gestão:} Dotar a plataforma de um painel de administração para a gestão de múltiplos projetos e utilizadores (anotadores e administradores), com controlo de acesso.
    
    \item \textbf{Automatizar o Cálculo de Métricas de Qualidade:} Implementar o cálculo automático do \textit{Inter-Annotator Agreement} (IAA) como uma funcionalidade central, permitindo a análise da consistência entre anotadores diretamente na plataforma.
    
    \item \textbf{Garantir a Interoperabilidade dos Dados:} Assegurar que os dados (mensagens e anotações) possam ser facilmente importados e exportados em formatos padrão (CSV e JSON), facilitando a integração da ferramenta em fluxos de trabalho de investigação mais vastos.
\end{enumerate}

A principal contribuição deste trabalho é a entrega de uma ferramenta de anotação \textit{open-source}, funcional e especializada, que preenche a lacuna identificada e suporta um fluxo de trabalho de anotação completo e integrado.

\section{Estrutura do Documento}

Este documento está organizado da seguinte forma:

\begin{itemize}
    \item \textbf{Capítulo 2 - Pertinência e Viabilidade}: Apresenta uma análise do estado da arte e compara a solução proposta com ferramentas existentes, justificando a sua relevância.

    \item \textbf{Capítulo 3 - Especificação e Modelação}: Detalha os requisitos funcionais e não-funcionais, os casos de uso e a modelação da arquitetura e da base de dados.

    \item \textbf{Capítulo 4 - Solução Proposta}: Descreve em detalhe a arquitetura final da aplicação, as tecnologias utilizadas e os seus principais componentes.

    \item \textbf{Capítulo 5 - Testes e Validação}: Apresenta a estratégia de verificação técnica utilizada para garantir a robustez e o correto funcionamento do sistema.

    \item \textbf{Capítulo 6 - Método e Planeamento}: Detalha a metodologia de desenvolvimento e realiza uma análise crítica do planeamento face à execução real do projeto.

    \item \textbf{Capítulo 7 - Resultados}: Apresenta os resultados concretos da implementação, incluindo as funcionalidades da plataforma, o fluxo de trabalho suportado e a documentação técnica gerada.

    \item \textbf{Capítulo 8 - Conclusão}: Sintetiza as contribuições do trabalho, discute as suas limitações e aponta direções para trabalhos futuros.
\end{itemize}
