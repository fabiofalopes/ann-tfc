\chapter{Resultados}
\label{cha:resultados}

Este capítulo apresenta os resultados concretos obtidos com o desenvolvimento da ferramenta de anotação. A apresentação está dividida em três áreas principais: uma visão geral da plataforma final, uma análise detalhada das funcionalidades chave implementadas, com especial destaque para o cálculo automático de métricas de concordância, e a documentação técnica produzida.

\section{Apresentação da Plataforma}

A solução desenvolvida é uma aplicação web completa que serve como um ambiente integrado para a anotação de chatrooms, especificamente para a tarefa de \textit{chat disentanglement}. A plataforma foi desenhada para servir dois perfis de utilizador distintos: o \textbf{Anotador}, focado na tarefa de anotação, e o \textbf{Administrador}, responsável pela gestão de projetos, utilizadores e pela análise dos dados.

O fluxo de utilização da aplicação segue um percurso lógico:
\begin{enumerate}
    \item \textbf{Autenticação:} O utilizador acede através de uma página de Login.
    \item \textbf{Dashboard:} Após o login, é apresentado um dashboard adaptado ao seu perfil.
    \begin{itemize}
        \item \textbf{Anotador:} Vê uma lista dos projetos a que está atribuído (\texttt{AnnotatorDashboard}). Ao selecionar um projeto, vê as salas de chat disponíveis (\texttt{AnnotatorProjectPage}).
        \item \textbf{Administrador:} Vê uma visão geral de todos os projetos e utilizadores no sistema (\texttt{AdminDashboard}).
    \end{itemize}
    \item \textbf{Anotação:} O anotador seleciona uma sala de chat e entra na interface de anotação (\texttt{AnnotatorChatRoomPage}), onde pode visualizar as mensagens e atribuir-lhes \textit{thread IDs}.
    \item \textbf{Análise (Admin):} O administrador pode aceder a uma página de projeto (\texttt{AdminProjectPage}) para gerir atribuições, importar dados e, crucialmente, visualizar a análise de concordância entre anotadores (\texttt{AdminChatRoomView}).
\end{enumerate}
% TODO: Esta secção deve ser acompanhada por um diagrama de navegação e screenshots das principais páginas, utilizando os recursos disponíveis em `sitemap_screenshots` para ilustrar o percurso do utilizador.

\section{Resultados da Implementação}

Esta secção detalha as funcionalidades mais relevantes que foram implementadas, demonstrando o cumprimento dos requisitos definidos e respondendo ao feedback recebido pelo júri.

\subsection{Cálculo de Inter-Annotator Agreement (IAA)}

Uma das funcionalidades centrais da plataforma é o cálculo automático da métrica de concordância entre anotadores (IAA), um requisito explícito do júri. A plataforma implementa o algoritmo \textbf{"1-to-1 agreement"}, que avalia a concordância estrutural entre os \textit{threads} criados por diferentes anotadores, focando-se no conteúdo das conversas e não nos nomes arbitrários dos \textit{threads}.

O processo de cálculo, disponível na visão do administrador para cada sala de chat, é o seguinte:
\begin{enumerate}
    \item \textbf{Agregação de Anotações:} Para uma dada sala de chat, o sistema primeiro agrupa todas as anotações por cada anotador. O resultado é um conjunto de "documentos de anotação", um para cada utilizador, onde cada documento contém os \textit{threads} que esse utilizador definiu.
    \item \textbf{Cálculo Par a Par (Pairwise):} A concordância é então calculada para cada par de anotadores possível. Por exemplo, numa sala com três anotadores (A, B, C), o sistema calcula o IAA para (A, B), (A, C), e (B, C).
    \item \textbf{Média Global:} O valor final de IAA apresentado para a sala de chat é a média aritmética de todos os scores de concordância calculados entre os pares.
\end{enumerate}

O núcleo do algoritmo (\texttt{\_calculate\_one\_to\_one\_accuracy} em \texttt{crud.py}) resolve um problema de atribuição. Para cada par de anotadores, ele constrói uma matriz de custo onde a dissimilaridade entre um \textit{thread} do Anotador 1 e um \textit{thread} do Anotador 2 é calculada com base no \textbf{Índice de Jaccard}. O índice mede a sobreposição de mensagens entre os dois \textit{threads}:
\[ J(A, B) = \frac{|A \cap B|}{|A \cup B|} \]
A dissimilaridade na matriz é, portanto, \(1 - J(A, B)\). Com esta matriz, o \textbf{Algoritmo Húngaro} (via \texttt{scipy.optimize.linear\_sum\_assignment}) encontra a correspondência ótima "um-para-um" entre os \textit{threads} que maximiza a semelhança total.

O resultado é uma matriz de similaridade, apresentada na UI do administrador, e um score de concordância final.
% TODO: Incluir uma figura do ecrã de Análise de IAA que mostra a matriz e o score final, explicando como um administrador a interpretaria.

\subsection{Gestão de Projetos e Utilizadores}

A plataforma fornece uma interface de administração robusta que permite a gestão completa do ciclo de vida de um projeto de anotação. As funcionalidades implementadas, acessíveis apenas ao administrador, incluem:
\begin{itemize}
    \item \textbf{Gestão de Projetos:} Criação, listagem e remoção de projetos.
    \item \textbf{Gestão de Utilizadores:} Criação, listagem e remoção de utilizadores.
    \item \textbf{Atribuição a Projetos:} Atribuição granular de utilizadores a projetos específicos, o que garante o isolamento dos dados e a correta distribuição de tarefas.
\end{itemize}

\subsection{Importação e Exportação de Dados}

Para facilitar a integração com outros fluxos de trabalho, foram desenvolvidas funcionalidades de importação e exportação de dados:
\begin{itemize}
    \item \textbf{Importação de Mensagens:} Os administradores podem iniciar um projeto importando uma chatroom completo a partir de um ficheiro CSV.
    \item \textbf{Importação de Anotações:} O sistema suporta a importação em lote de anotações a partir de um ficheiro JSON, que pode conter anotações de múltiplos utilizadores.
    \item \textbf{Exportação de Projetos:} Todos os dados de uma sala de chat (mensagens e a totalidade das anotações) podem ser exportados para um único ficheiro JSON.
\end{itemize}

\section{Documentação Técnica da API}

Uma vantagem inerente à escolha da framework FastAPI é a geração automática de uma especificação da API que segue o standard \textbf{OpenAPI 3.0}. Este processo resulta num ficheiro \texttt{openapi.json} que descreve todos os endpoints da aplicação, os seus parâmetros e os formatos de dados esperados.

O principal objetivo prático deste ficheiro no contexto do projeto foi facilitar os testes e a validação do backend. Ao importar esta especificação para ferramentas de desenvolvimento como o Postman, foi possível testar cada endpoint de forma sistemática e eficiente durante o ciclo de desenvolvimento, garantindo o seu correto funcionamento.

% TODO: Inserir uma pequena tabela ou imagem que resuma os principais grupos de endpoints da API: Auth, Projects, Annotations e Admin, para dar uma visão geral da sua estrutura.
