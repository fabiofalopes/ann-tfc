\chapter{Resultados}
\label{cha:resultados}

Este capítulo apresenta os resultados concretos obtidos com o desenvolvimento da ferramenta de anotação. O conteúdo está organizado em três áreas principais: uma visão geral da ferramenta final, uma análise detalhada das funcionalidades chave implementadas, com especial destaque para o cálculo automático de métricas de concordância, e a documentação técnica produzida.

\section{Apresentação da Ferramenta}

A solução desenvolvida é uma aplicação web completa que serve como um ambiente integrado para a anotação de chatrooms, especificamente para a tarefa de \textit{chat disentanglement}. A ferramenta foi concebida para servir dois perfis de utilizador distintos: o \textbf{Anotador}, focado na tarefa de anotação, e o \textbf{Administrador}, responsável pela gestão de projetos, utilizadores e pela análise dos dados.

\subsection{Fluxo de Trabalho Suportado}

Uma das principais contribuições desta ferramenta é a definição e implementação de um fluxo de trabalho estruturado para anotação de \textit{chat disentanglement}. Embora existam diversas ferramentas de anotação disponíveis, muitas com funcionalidades extensas, a proposta diferenciadora desta solução é a integração de um fluxo de trabalho completo e especializado.

O fluxo de trabalho implementado segue um percurso lógico em 6 etapas principais:

\begin{enumerate}
    \item \textbf{Configuração de Projeto:} O administrador cria um novo projeto de anotação e define os utilizadores participantes.
    \item \textbf{Importação de Dados:} Os dados de conversas são importados através de ficheiros CSV, estruturando-os na base de dados.
    \item \textbf{Atribuição de Tarefas:} Os anotadores são atribuídos a projetos específicos, garantindo isolamento e distribuição adequada das tarefas.
    \item \textbf{Processo de Anotação:} Os anotadores acedem às suas tarefas através de dashboards personalizados, selecionam salas de chat e realizam a anotação através de uma interface especializada.
    \item \textbf{Análise de Concordância:} O sistema calcula automaticamente métricas de concordância inter-anotador (IAA), fornecendo feedback sobre a qualidade das anotações.
    \item \textbf{Exportação de Resultados:} Os dados anotados são exportados em formato JSON para análise posterior ou integração com outros sistemas.
\end{enumerate}

\subsection{Navegação e Interface}

O percurso de utilização da aplicação implementa este fluxo de trabalho através das seguintes interfaces:

\begin{enumerate}
    \item \textbf{Autenticação:} O utilizador acede através de uma página de Login.
    \item \textbf{Dashboard:} Após o login, é apresentado um dashboard adaptado ao seu perfil.
    \begin{itemize}
        \item \textbf{Anotador:} Vê uma lista dos projetos a que está atribuído (\texttt{AnnotatorDashboard}). Ao selecionar um projeto, vê as salas de chat disponíveis (\texttt{AnnotatorProjectPage}).
        \item \textbf{Administrador:} Vê uma visão geral de projetos e utilizadores no sistema (\texttt{AdminDashboard}).
    \end{itemize}
    \item \textbf{Anotação:} O anotador seleciona uma sala de chat e entra na interface de anotação (\texttt{AnnotatorChatRoomPage}), onde pode visualizar as mensagens e atribuir-lhes a respetiva anotação (\textit{thread IDs}).
    \item \textbf{Análise (Admin):} O administrador pode aceder a uma página de projeto (\texttt{AdminProjectPage}) para gerir atribuições, importar dados e, crucialmente, visualizar a análise de concordância entre anotadores (\texttt{AdminChatRoomView}).
\end{enumerate}

\section{Resultados da Implementação}

Esta secção detalha as funcionalidades mais relevantes que foram implementadas, demonstrando o cumprimento dos requisitos definidos e respondendo ao feedback recebido, previamente, pelo juri.

\subsection{Cálculo de Inter-Annotator Agreement (IAA)}

Uma das funcionalidades centrais da ferramenta é o cálculo automático da métrica de concordância entre anotadores (IAA), um requisito explícito identificado durante o desenvolvimento. A ferramenta implementa o algoritmo \textbf{"1-to-1 agreement"} \cite{elsner2008you}, que avalia a concordância estrutural entre os \textit{threads} criados por diferentes anotadores, focando-se no indice de cada turno/mensagem anotada, e não nos labels das \textit{threads}.

O processo de cálculo, disponível na visão do administrador para cada sala de chat, é o seguinte:
\begin{enumerate}
    \item \textbf{Agregação de Anotações:} Para uma dada sala de chat, o sistema agrupa as anotações por cada anotador. O resultado é um conjunto de "documentos de anotação", um para cada utilizador, onde cada documento contém os \textit{threads} que esse utilizador definiu.
    \item \textbf{Cálculo Par a Par (Pairwise):} A concordância é então calculada para cada par de anotadores possível. Por exemplo, numa sala com três anotadores (A, B, C), o sistema calcula o IAA para (A, B), (A, C), e (B, C).
    \item \textbf{Média Global:} O valor final de IAA apresentado para a sala de chat é a média aritmética dos valores de concordância calculados entre os pares.
\end{enumerate}

O núcleo do algoritmo (\texttt{\_calculate\_one\_to\_one\_accuracy} em \texttt{crud.py}) resolve um problema de atribuição. Para cada par de anotadores, ele constrói uma matriz de custo onde a dissimilaridade entre um \textit{thread} do Anotador 1 e um \textit{thread} do Anotador 2 é calculada com base no \textbf{Índice de Jaccard}. O índice mede a sobreposição de mensagens entre os dois \textit{threads}:
\[ J(A, B) = \frac{|A \cap B|}{|A \cup B|} \]
A dissimilaridade na matriz é, portanto, \(1 - J(A, B)\). Com esta matriz, o \textbf{Algoritmo Húngaro} (via \texttt{scipy.optimize.linear\_sum\_assignment}) encontra a correspondência ótima "um-para-um" entre os \textit{threads} que maximiza a semelhança total.

O resultado é uma matriz de similaridade, apresentada na UI do administrador, e um score de concordância final.

\subsection{Gestão de Projetos e Utilizadores}

A ferramenta fornece uma interface de administração robusta que permite a gestão completa do ciclo de vida de um projeto de anotação. As funcionalidades implementadas, acessíveis apenas ao administrador, incluem:
\begin{itemize}
    \item \textbf{Gestão de Projetos:} Criação, listagem e remoção de projetos.
    \item \textbf{Gestão de Utilizadores:} Criação, listagem e remoção de utilizadores.
    \item \textbf{Atribuição a Projetos:} Atribuição granular de utilizadores a projetos específicos, o que garante o isolamento dos dados e a correta distribuição de tarefas.
\end{itemize}

\subsection{Importação e Exportação de Dados}

Para facilitar a integração com outros fluxos de trabalho, foram desenvolvidas funcionalidades de importação e exportação de dados:
\begin{itemize}
    \item \textbf{Importação de Mensagens:} Os administradores podem iniciar um projeto importando uma chatroom completa a partir de um ficheiro CSV.
    \item \textbf{Importação de Anotações:} O sistema suporta a importação em lote de anotações a partir de um ficheiro JSON, que pode conter anotações de múltiplos utilizadores.
    \item \textbf{Exportação de Projetos:} Os dados de uma sala de chat (mensagens e anotações) podem ser exportados para um único ficheiro JSON.
\end{itemize}

\section{Contribuição Principal: Fluxo de Trabalho Integrado}

A principal contribuição desta ferramenta não reside apenas nas funcionalidades individuais implementadas, mas na integração dessas funcionalidades num fluxo de trabalho coeso e especializado para \textit{chat disentanglement}. Esta abordagem distingue-se de ferramentas genéricas de anotação por oferecer:

\begin{itemize}
    \item \textbf{Especialização:} Interface e funcionalidades otimizadas especificamente para a tarefa de disentanglement.
    \item \textbf{Automatização:} Cálculo automático de métricas de concordância, eliminando processamento manual posterior.
    \item \textbf{Gestão Integrada:} Desde a importação de dados até à exportação de resultados, o processo é gerido numa única ferramenta.
    \item \textbf{Controlo de Qualidade:} Métricas de concordância integradas no fluxo de trabalho para avaliação contínua da qualidade das anotações.
\end{itemize}

Esta integração permite que equipas de investigação tenham uma solução completa para projetos de anotação de \textit{chat disentanglement}, reduzindo a necessidade de ferramentas múltiplas e processos manuais.

\section{Documentação Técnica da API}

Uma vantagem inerente à escolha da framework FastAPI é a geração automática de uma especificação da API que segue o standard \textbf{OpenAPI 3.0}. Este processo resulta num ficheiro \texttt{openapi.json} que descreve os endpoints da aplicação, os seus parâmetros e os formatos de dados esperados.

O principal objetivo prático deste ficheiro no contexto do projeto foi facilitar os testes e a validação do backend. Ao importar esta especificação para ferramentas de desenvolvimento como o Postman, foi possível testar cada endpoint de forma sistemática e eficiente durante o ciclo de desenvolvimento, garantindo o seu correto funcionamento.
