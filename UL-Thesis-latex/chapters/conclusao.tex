\chapter{Conclusão}
\label{cha:conclusao}

\section{Conclusão}

O presente Trabalho Final de Curso teve como objetivo central o desenvolvimento de uma ferramenta de software especializada para a tarefa de anotação de \textit{chat disentanglement}. O problema de base identificado foi a ausência de plataformas dedicadas que não só facilitassem o processo de anotação manual, mas que também integrassem mecanismos de análise de qualidade e concordância, forçando frequentemente os investigadores a recorrer a ferramentas genéricas e a processos de cálculo de métricas desligados da tarefa principal.

Em resposta a este desafio, foi concebida, desenvolvida e implementada uma aplicação web completa. A solução, construída sobre uma arquitetura moderna cliente-servidor (React e FastAPI), materializa-se numa plataforma funcional que permite a gestão de projetos de anotação, a atribuição de tarefas a múltiplos anotadores e, mais importante, oferece uma interface otimizada para a tarefa de \textit{chat disentanglement}.

\subsection{Pipeline de Anotação Implementado}

A ferramenta implementa uma arquitetura que suporta o ciclo de vida da anotação através de uma API REST e interfaces especializadas. O sistema integra as seguintes funcionalidades:

\textbf{Ingestão e Processamento de Dados:} A aplicação oferece múltiplas vias para importação de dados de conversas. Através de endpoints REST específicos (\texttt{/api/projects/\{id\}/import-\linebreak messages}), a ferramenta processa ficheiros CSV contendo mensagens de chat. O sistema realiza validação, estruturação e persistência automática na base de dados SQLite. Esta funcionalidade suporta importação de grandes volumes de dados através de \textit{scripting} ou integração programática. Isto permite que equipas de investigação automatizem a ingestão de \textit{datasets} extensos sem intervenção manual.

\textbf{Interface de Anotação Especializada:} O frontend React implementa uma interface otimizada especificamente para \textit{chat disentanglement}. As funcionalidades incluem visualização cronológica de mensagens, atribuição interativa de \textit{thread IDs}, e gestão de estado em tempo real. A interface comunica com a API através de endpoints dedicados (\texttt{/api/annotations}) que garantem persistência imediata das anotações e sincronização entre múltiplos anotadores.

\textbf{Sistema de Análise Automática:} A ferramenta integra capacidades de análise quantitativa através da implementação do algoritmo "1-to-1 agreement" \cite{elsner2008you}. Este sistema processa automaticamente as anotações de múltiplos utilizadores, calculando métricas de concordância através de técnicas de otimização (Algoritmo Húngaro) e apresentando os resultados através de matrizes de similaridade e scores agregados. O cálculo é acionado dinamicamente através da API (\texttt{/api/admin/chatrooms/\{id\}/iaa}) e apresentado em tempo real na interface administrativa.

\textbf{Exportação e Integração:} A plataforma oferece capacidades de exportação estruturada através de endpoints REST que serializam dados de projetos completos em formato JSON. Esta funcionalidade permite integração com ferramentas de análise externa, facilitando a continuidade do trabalho de investigação e a interoperabilidade com outros sistemas de processamento de linguagem natural.

O principal resultado deste trabalho é uma ferramenta que cumpre os seus requisitos fundamentais. Destaca-se a implementação do cálculo automático do Inter-Annotator Agreement (IAA) através do algoritmo "1-to-1 agreement" \cite{elsner2008you}, que utiliza o Índice de Jaccard e o Algoritmo Húngaro para fornecer uma medida robusta da concordância estrutural entre anotadores. Esta funcionalidade, que foi uma indicação explícita dos coordenadores do TFC, transforma a plataforma de uma simples ferramenta de anotação num ambiente de análise, permitindo aos gestores de projeto aferir a qualidade e a consistência das anotações diretamente no sistema.

No entanto, é com rigor académico que se reconhecem as limitações do trabalho realizado. A principal limitação reside na ausência de uma fase de validação formal com utilizadores finais. Embora a ferramenta seja funcional e tecnicamente robusta, não foi conduzido um estudo empírico para avaliar o seu impacto real na qualidade das anotações ou na experiência do anotador. Adicionalmente, alguns requisitos não-funcionais, como a implementação de backups automáticos e testes de carga formais, foram considerados fora do âmbito da fase de desenvolvimento atual.

\section{Trabalhos Futuros}

As limitações identificadas abrem um caminho claro para trabalhos futuros, que poderiam elevar significativamente o impacto e a maturidade do projeto:

\begin{itemize}
    \item \textbf{Estudo de Validação com Utilizadores:} O passo mais crítico seria a realização de um estudo formal. Este estudo envolveria um grupo de anotadores que realizaria a mesma tarefa de anotação utilizando a nossa ferramenta e um método de base (e.g., folha de cálculo), analisando métricas objetivas (tempo, scores de IAA) e subjetivas (inquéritos de satisfação e usabilidade).
    \item \textbf{Expansão de Métricas de Análise:} A plataforma poderia ser enriquecida com o cálculo de outras métricas de concordância, como o Krippendorff's Alpha, que é mais flexível em cenários com múltiplos anotadores e dados em falta.
    \item \textbf{Generalização da Ferramenta:} A arquitetura foi pensada de forma modular. Um próximo passo seria abstrair o processo de anotação para que a plataforma pudesse ser configurada para outros tipos de tarefas (e.g., anotação de entidades, análise de sentimento).
    \item \textbf{Melhorias de Infraestrutura e Performance:} Implementar as funcionalidades de backup e realizar testes de carga para otimizar o desempenho da API e da base de dados para um número elevado de utilizadores concorrentes, incluindo a migração para um driver de base de dados assíncrono.
\end{itemize}

Em suma, este projeto entregou com sucesso uma solução aplicacional concreta para um problema real no domínio do processamento de linguagem natural, estabelecendo uma base sólida sobre a qual futuras investigações e desenvolvimentos podem ser construídos.
