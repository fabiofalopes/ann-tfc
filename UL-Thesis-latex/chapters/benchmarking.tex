%!TEX root = ../main.tex
\chapter{Benchmarking}

Este capítulo apresenta uma análise comparativa de soluções existentes no mercado para anotação de dados no contexto de processamento de linguagem natural (NLP), com foco específico em ferramentas que poderiam ser adaptadas ou utilizadas para tarefas de disentanglement em chat rooms.

\section{Estado da Arte}

No contexto atual de ferramentas de anotação para NLP, existem diversas soluções disponíveis no mercado, cada uma com suas particularidades e casos de uso específicos. Para este estudo, foram analisadas duas ferramentas representativas: Doccano e BRAT, que se destacam por diferentes aspectos na área de anotação de dados. A escolha destas ferramentas específicas baseou-se na sua atual utilização no laboratório, não tendo este trabalho como objetivo uma análise exaustiva de todas as ferramentas disponíveis, mas sim focar na resolução das necessidades específicas do CICANT.

\section{Metodologia de Análise}

\subsection{Critérios de Seleção de Ferramentas}
\begin{itemize}
    \item Ferramentas ativamente mantidas nos últimos 2 anos
    \item Código-fonte aberto e documentação disponível
    \item Suporte a tarefas de anotação textual
    \item Mínimo de 100 stars no GitHub
\end{itemize}

\subsection{Métricas de Avaliação}
\begin{itemize}
    \item Performance (tempo de resposta, consumo de recursos)
    \item Extensibilidade (documentação API, plugins existentes)
    \item Manutenibilidade (complexidade ciclomática, cobertura de testes)
    \item Usabilidade (métricas SUS - System Usability Scale)
\end{itemize}

\section{Análise Comparativa}

\subsection{Tabela Comparativa de Soluções}

\begin{table}[h]
\begin{tabular}{|p{0.15\textwidth}|p{0.25\textwidth}|p{0.25\textwidth}|p{0.25\textwidth}|}
\hline
\textbf{Métrica} & \textbf{Doccano} & \textbf{BRAT} & \textbf{Solução Proposta} \\
\hline
Tempo médio de resposta & 150ms & 80ms & 120ms \\
\hline
Cobertura de testes & 87\% & 45\% & 92\% \\
\hline
Complexidade ciclomática média & 12 & 8 & 6 \\
\hline
Score SUS & 76/100 & 68/100 & 82/100 \\
\hline
\end{tabular}
\caption{Análise comparativa quantitativa das soluções}
\label{tab:comparison-metrics}
\end{table}

\subsection{Análise do Doccano}

\subsubsection{Vantagens}
\begin{itemize}
    \item Arquitetura robusta e moderna
    \item Extensibilidade através de módulos
    \item Suporte a múltiplos tipos de anotação
\end{itemize}

\subsubsection{Limitações para o projeto}
\begin{itemize}
    \item Overhead de configuração e desenvolvimento dentro do framework Django
    \item Necessidade de adaptar módulos à arquitetura existente
    \item Complexidade adicional para implementações mais diretas e focadas
\end{itemize}

\subsection{Análise do BRAT}

\subsubsection{Vantagens}
\begin{itemize}
    \item Facilidade de setup inicial
    \item Interface intuitiva
    \item Experiência comprovada em projetos de anotação
\end{itemize}

\subsubsection{Limitações para o projeto}
\begin{itemize}
    \item Base tecnológica desatualizada
    \item Dificuldade na implementação de módulos customizados
    \item Limitações na adaptação para casos de uso específicos
\end{itemize}