%!TEX root = ../main.tex
\chapter{Testes e Validação}

\section{Introdução}

Este capítulo descreve como a ferramenta de anotação foi testada e validada. O objetivo é verificar se a ferramenta funciona corretamente, se é adequada para a tarefa de \textit{chat disentanglement}, e se representa uma melhoria face a métodos de anotação mais manuais. A validação pretende confirmar a aplicabilidade prática da solução desenvolvida e demonstrar o cumprimento dos objetivos propostos, nomeadamente o de contribuir para a solução de um problema real de investigação.

\section{Abordagem e Justificação dos Testes}

\subsection{Estratégia de Validação}
A validação da ferramenta baseou-se numa abordagem de avaliação técnica interna realizada em conjunto com a equipa do laboratório AISIC. Esta estratégia foi escolhida por permitir:
\begin{itemize}
    \item Uma análise técnica detalhada das funcionalidades implementadas.
    \item Avaliação por utilizadores com conhecimento especializado na área de processamento de linguagem natural.
    \item Validação em contexto real de investigação, com dados e cenários autênticos.
    \item Identificação de limitações e áreas de melhoria numa fase adequada do desenvolvimento.
\end{itemize}

\subsection{Critérios de Aceitação}
Os testes foram estruturados para validar o cumprimento dos seguintes critérios de aceitação:
\begin{itemize}
    \item \textbf{Funcionalidade Técnica:} A ferramenta deve importar, processar e exportar dados de conversas corretamente.
    \item \textbf{Adequação à Tarefa:} A interface deve facilitar a realização de tarefas de \textit{chat disentanglement}.
    \item \textbf{Cálculo de Métricas:} O sistema deve calcular automaticamente métricas de concordância inter-anotador.
    \item \textbf{Gestão de Projetos:} O sistema deve permitir gestão adequada de utilizadores e projetos de anotação.
    \item \textbf{Eficiência Operacional:} A ferramenta deve representar uma melhoria face a métodos manuais.
\end{itemize}

\subsection{Participantes e Perfis de Avaliação}
A avaliação envolveu dois perfis principais de investigadores:
\begin{itemize}
    \item \textbf{Administradores:} Investigadores familiarizados com a gestão de sistemas e projetos de investigação, que avaliaram as funcionalidades administrativas da ferramenta.
    \item \textbf{Anotadores:} Um grupo mais alargado de participantes (investigadores, estudantes) com experiência ou interesse na análise de dados textuais, que se focaram na tarefa central de anotação para \textit{chat disentanglement}.
\end{itemize}

\section{Cenários e Tarefas de Validação}

As tarefas de validação procuraram simular o ciclo de vida completo da utilização da ferramenta:

\begin{itemize}
    \item \textbf{Tarefas de Administração:} Avaliadas pelos investigadores com perfil administrativo, incluíram:
    \begin{itemize}
        \item Criação e gestão de projetos de anotação.
        \item Gestão de utilizadores (criação, atribuição de papéis e permissões).
        \item Importação de dados (ficheiros de \textit{chat}) para os projetos.
        \item Verificação da correta importação e organização dos dados.
    \end{itemize}
    \item \textbf{Tarefas de Anotação:} Avaliadas pelos investigadores com experiência em anotação, incluíram:
    \begin{itemize}
        \item Navegação e visualização das conversas importadas.
        \item Utilização da interface para identificar e marcar diferentes \textit{threads}.
        \item Atribuição de mensagens às \textit{threads} correspondentes.
        \item Edição e correção de anotações realizadas.
        \item Exportação das anotações finalizadas.
    \end{itemize}
\end{itemize}

\section{Validação Técnica e de Recursos}

\subsection{Validação da Infraestrutura Técnica}
Foi realizada uma validação detalhada dos recursos computacionais e técnicos da solução:
\begin{itemize}
    \item \textbf{Armazenamento:} Validação da capacidade de armazenamento de dados na base SQLite, incluindo testes com conjuntos de dados de diferentes dimensões.
    \item \textbf{Desempenho:} Avaliação dos tempos de resposta para operações críticas (importação, cálculo de métricas, exportação).
    \item \textbf{Estabilidade:} Testes de funcionamento contínuo e robustez da aplicação durante sessões prolongadas de anotação.
    \item \textbf{Compatibilidade:} Verificação do funcionamento em diferentes sistemas operativos e navegadores web.
\end{itemize}

\subsection{Validação da Importação e Gestão de Dados}
Foi realizada uma validação técnica detalhada dos processos de importação de dados, verificando:
\begin{itemize}
    \item Correta estruturação e organização dos dados na base de dados.
    \item Integridade dos dados durante o processo de importação.
    \item Funcionalidade de exportação dos dados anotados em formatos apropriados para análise posterior.
\end{itemize}

\subsection{Avaliação da Interface de Anotação}
A interface de anotação foi avaliada em reunião pela equipa do laboratório, focando:
\begin{itemize}
    \item Adequação da visualização das conversas para a tarefa de disentanglement.
    \item Facilidade de identificação e marcação de diferentes \textit{threads}.
    \item Eficiência do processo de atribuição de mensagens a \textit{threads}.
    \item Capacidade de edição e correção de anotações.
    \item Clareza visual da organização das anotações realizadas.
\end{itemize}

\subsection{Validação das Métricas de Concordância}
Uma funcionalidade central da ferramenta é o cálculo automático de métricas de concordância inter-anotador (IAA - \textit{Inter-Annotator Agreement}). Foi validado:
\begin{itemize}
    \item Correto funcionamento do algoritmo de cálculo de concordância "one-to-one" \cite{elsner2008you}.
    \item Adequação das métricas implementadas para a avaliação da qualidade das anotações.
    \item Capacidade de comparação entre múltiplos anotadores no mesmo conjunto de dados.
    \item Funcionalidade de exportação das métricas calculadas para análise externa.
\end{itemize}

\section{Resultados da Validação}

\subsection{Cumprimento dos Critérios de Aceitação}
A validação confirmou o cumprimento de todos os critérios de aceitação estabelecidos:

\begin{itemize}
    \item \textbf{Funcionalidade Técnica:} Os processos de importação e exportação de dados funcionam de forma robusta.
    \item \textbf{Adequação à Tarefa:} A interface de anotação permite realizar eficientemente as tarefas de \textit{chat disentanglement}.
    \item \textbf{Cálculo de Métricas:} As métricas de concordância são calculadas automaticamente e fornecem informação útil sobre a qualidade das anotações.
    \item \textbf{Gestão de Projetos:} O sistema de gestão de utilizadores e projetos adequa-se às necessidades de um ambiente de investigação.
    \item \textbf{Eficiência Operacional:} Confirmada melhoria significativa face a métodos manuais.
\end{itemize}

\subsection{Validação de Recursos Técnicos}
Os testes de infraestrutura confirmaram:
\begin{itemize}
    \item Desempenho adequado para conjuntos de dados típicos de investigação.
    \item Compatibilidade com os principais navegadores web e sistemas operativos.
    \item Capacidade de armazenamento suficiente para múltiplos projetos simultâneos.
\end{itemize}

\subsection{Adequação ao Fluxo de Trabalho}
A avaliação em reunião pela equipa do laboratório indicou que as tarefas de anotação foram avaliadas como mais eficientes quando realizadas com a ferramenta, comparativamente a abordagens manuais ou anotações realizadas em outras ferramentas citadas neste trabalho. Especificamente:
\begin{itemize}
    \item A visualização estruturada das conversas facilita a identificação de \textit{threads}.
    \item O processo de atribuição de mensagens é mais intuitivo que métodos totalmente manuais.
    \item O cálculo automático de métricas de concordância elimina a necessidade de processamento manual posterior.
    \item A capacidade de exportação permite integração com fluxos de trabalho de análise existentes.
\end{itemize}

\subsection{Limitações Identificadas}
Durante a validação, foram também identificadas algumas limitações:
\begin{itemize}
    \item A ferramenta é específica para tarefas de \textit{chat disentanglement}, não sendo diretamente aplicável a outros tipos de anotação.
    \item A interface assume familiaridade básica com conceitos de anotação de dados textuais.
\end{itemize}

\section{Conclusões da Validação}

A validação realizada confirma que a ferramenta desenvolvida cumpre os objetivos propostos, fornecendo uma solução técnica adequada para tarefas de \textit{chat disentanglement}. Todos os critérios de aceitação foram cumpridos, e a avaliação pela equipa do laboratório indica que a ferramenta representa uma melhoria significativa face a métodos manuais.

A validação técnica de recursos confirmou a adequação da infraestrutura para o contexto de investigação, com desempenho e estabilidade apropriados para as necessidades identificadas. A automatização do cálculo de métricas de concordância e a estruturação do processo de anotação constituem as principais contribuições para a melhoria do fluxo de trabalho.

Embora reconhecendo as limitações inerentes a uma validação técnica interna e a especificidade da ferramenta para \textit{chat disentanglement}, os resultados fornecem indicações sobre a pertinência e eficácia da solução desenvolvida, confirmando a sua adequação ao contexto de investigação em processamento de linguagem natural e análise de conversas.
