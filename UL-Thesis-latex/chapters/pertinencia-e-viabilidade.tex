\chapter{Pertinência e Viabilidade}

\section{Pertinência}
O desenvolvimento de uma ferramenta modular para anotação de dados surge como resposta a uma necessidade crítica no campo do processamento de linguagem natural (NLP). A pertinência desta solução é evidenciada por múltiplos fatores.

O AISIC LAB, validou a necessidade desta ferramenta através de feedback direto dos anotadores sobre as limitações das ferramentas atuais, bem como através da avaliação dos investigadores sobre o impacto na qualidade dos dados e análise das necessidades específicas de projetos em andamento.

O desenvolvimento desta ferramenta promete uma redução significativa no tempo de anotação, além de proporcionar uma melhoria substancial na qualidade e consistência dos dados anotados. A solução também facilita a colaboração entre anotadores e oferece suporte flexível a múltiplos tipos de tarefas de anotação, adaptando-se às necessidades específicas de cada projeto.

\section{Viabilidade}
A implementação da solução demonstra-se viável em múltiplas dimensões. Do ponto de vista técnico, a experiência prévia com um protótipo funcional em React, combinada com a disponibilidade de frameworks modernos para desenvolvimento, fornece uma base sólida para o projeto. A arquitetura modular planeada permite um desenvolvimento incremental e sustentável, aproveitando a infraestrutura existente para deployment.

Quanto à viabilidade económica, o projeto beneficia de um baixo custo de desenvolvimento inicial, principalmente devido à utilização de tecnologias open-source. Além disso, apresenta potencial para comercialização futura e promete uma redução significativa nos custos operacionais relacionados à anotação de dados.

A aceitação social do projeto é confirmada pelo feedback positivo dos utilizadores finais e pelo forte alinhamento com as necessidades do departamento. O potencial de aplicação em outros contextos académicos e a contribuição significativa para a pesquisa em NLP reforçam sua relevância no ambiente académico e científico.

\section{Análise Comparativa com Soluções Existentes}

\subsection{Soluções Existentes}

\subsubsection{Doccano}
O Doccano é uma plataforma open-source para anotação de dados em NLP. A ferramenta oferece suporte a múltiplos tipos de anotação e possui uma arquitetura modular que permite extensões dentro da sua estrutura.

\subsubsection{BRAT}
O BRAT é uma ferramenta estabelecida para anotação de texto, com foco em simplicidade e interface intuitiva. Apesar de sua maturidade, apresenta limitações em termos de extensibilidade e modernização.

\subsection{Análise de Benchmarking}

A Tabela~\ref{tab:tech-comparison} apresenta uma comparação detalhada das características técnicas de cada solução. Esta análise permitiu identificar pontos fortes e limitações de cada ferramenta, orientando o desenvolvimento da solução proposta.

\section{Proposta de Inovação e Mais-Valias}
A solução desenvolvida foca-se especificamente nas necessidades de anotação para a tarefa dedisentanglement de chat nesta fase inicial, permitindo uma implementação direta e eficiente deste tipo de tarefa. A abordagem modular facilita a adaptação para diferentes necessidades de anotação, com o objetivo futuro de suportar múltiplos tipos de tarefas de anotação, mantendo a simplicidade de uso.

\section{Identificação de Oportunidade de Negócio}
O projeto apresenta potencial para aplicação em contextos acadêmicos e de pesquisa, especialmente em grupos que trabalham com processamento de linguagem natural. A capacidade de adaptação para diferentes tipos de anotação permite atender necessidades específicas de diversos projetos de pesquisa.

\clearpage
\begin{table}[p]
\setlength{\footnotesep}{15pt}
\begin{minipage}{\textwidth}
\centering
\begin{tabular}{|l|p{0.25\textwidth}|p{0.25\textwidth}|p{0.25\textwidth}|}
\hline
\textbf{Solução} & \textbf{BRAT\protect\footnotemark[1]} & \textbf{Doccano\protect\footnotemark[2]} & \textbf{Solução Proposta} \\
\hline
Frontend & jQuery 1.7.1 + jQuery UI, SVG para visualização, JavaScript vanilla, XHTML templates & Nuxt.js framework, Vue.js (implícito via Nuxt), Modern JavaScript/TypeScript & React 18, TypeScript, Componentes funcionais, Hooks customizados \\
\hline
Backend & Python 2.5+, CGI/FastCGI, Bibliotecas JSON & Python 3.8+, Django 4.0+, REST API architecture & Express.js (protótipo), Migração planeada para Flask/FastAPI \\
\hline
Formato de Anotação & Arquivos .ann para anotações, Arquivos .txt para texto, JSON para comunicação & REST API endpoints, JSON para comunicação & REST API para comunicação e configuração, Suporte planeado para múltiplos formatos \\
\hline
Dados & Sistema de arquivos, Estrutura em diretórios, Sem banco de dados & Database (Django ORM), Suporte a PostgreSQL & Suporte planeado para múltiplos formatos (CSV, TXT, JSON, Markdown), Cache local opcional via SQLite \\
\hline
Deployment & Apache/Lighttpd com CGI, Standalone Python server & Docker containers, Docker Compose support, Production/Development configs & Container Docker único, Setup simplificado \\
\hline
Extensibilidade & Sistema via arquivos .conf, Arquitetura modular, Plugins jQuery & Modular Django architecture, REST API extensibility, Frontend component system & Componentes React modulares, Hooks customizáveis, API extensível \\
\hline
Estado de Manutenção & Última atualização ~2012, Tecnologias desatualizadas, Projeto inativo & Projeto ativo, Tecnologias modernas, Atualizações regulares & Em desenvolvimento ativo, Stack moderna, Iterações frequentes \\
\hline
Documentação & README detalhado, Exemplos incluídos, Tutoriais no código & Documentação estruturada, Guias de desenvolvimento, Diagramas de arquitetura & Documentação focada, Exemplos práticos, Guias de desenvolvimento \\
\hline
Requisitos Sistema & Python 2.5+, Servidor Web com CGI, Browser com SVG & Python 3.8+, Docker (recomendado), Node.js para desenvolvimento & Node.js 18+ (atual), Python 3.x (planeado), Navegador moderno, Docker (opcional) \\
\hline
Segurança & Autenticação básica, Controle via arquivo & Django authentication, Modern security practices & Autenticação básica baseada em roles (administrador/anotador) \\
\hline
\end{tabular}
\caption{Comparação técnica detalhada das soluções}
\label{tab:tech-comparison}

\footnotetext[1]{\cite{brat-repo,brat-about}}
\footnotetext[2]{\cite{doccano-repo,doccano-docs}}
\end{minipage}
\end{table}

