\chapter{Pertinência e Viabilidade}

\section{Pertinência}
O desenvolvimento de uma ferramenta especializada para anotação de dados surge como resposta a uma necessidade crítica no campo do processamento de linguagem natural (NLP). A pertinência desta solução é evidenciada por múltiplos fatores.

O AISIC LAB validou a necessidade desta ferramenta através de feedback direto dos anotadores sobre as limitações das ferramentas atuais, bem como através da avaliação dos investigadores sobre o impacto na qualidade dos dados e análise das necessidades específicas de projetos em andamento.

O desenvolvimento desta ferramenta promete uma redução significativa no tempo de anotação, além de proporcionar uma melhoria substancial na qualidade e consistência dos dados anotados. A solução também facilita a colaboração entre anotadores e oferece funcionalidades específicas para a tarefa de \textit{chat disentanglement}, adaptando-se às necessidades identificadas no laboratório.

\section{Viabilidade}
A implementação da solução demonstra-se viável em múltiplas dimensões. Do ponto de vista técnico, a experiência prévia com um protótipo funcional em React, aliada à atual implementação utilizando FastAPI e SQLite, combinada com a disponibilidade de \textit{frameworks} modernos para desenvolvimento, fornece uma base sólida para o projeto. A arquitetura implementada permite um desenvolvimento incremental e sustentável, aproveitando a infraestrutura existente para \textit{deployment}.

Quanto à viabilidade económica, o projeto beneficia de um baixo custo de desenvolvimento inicial, principalmente devido à utilização de tecnologias \textit{open-source}. Além disso, apresenta potencial para utilização em outros contextos académicos e promete uma redução significativa nos custos operacionais relacionados à anotação de dados.

A aceitação social do projeto é confirmada pelo \textit{feedback} positivo dos orientadores e pelo forte alinhamento com as necessidades do departamento. O potencial de aplicação em outros contextos académicos e a contribuição significativa para a investigação em NLP reforçam sua relevância no ambiente académico e científico.

\section{Análise Comparativa com Soluções Existentes}

A análise comparativa apresentada foca-se nas ferramentas e métodos previamente apontados e utilizados no laboratório AISIC para tarefas similares de anotação, não constituindo uma análise exaustiva de todas as soluções disponíveis no mercado.

\subsection{Soluções Existentes}

\subsubsection{Doccano}
O Doccano é uma plataforma \textit{open-source} para anotação de dados em NLP. A ferramenta oferece suporte a múltiplos tipos de anotação e possui uma arquitetura baseada em Django que permite extensões dentro da sua estrutura.

\subsubsection{BRAT}
O BRAT é uma ferramenta estabelecida para anotação de texto, com foco em simplicidade e interface intuitiva. Apesar de sua maturidade, apresenta limitações em termos de extensibilidade e modernização tecnológica.

\subsubsection{Excel}
Folhas de cálculo representam uma abordagem manual frequentemente utilizada para tarefas de anotação. Embora acessível e familiar para muitos utilizadores, este método apresenta limitações significativas em termos de consistência, controlo de qualidade e cálculo automático de métricas de concordância entre anotadores.

\subsection{Análise de Benchmarking}

A Tabela~\ref{tab:tech-comparison} apresenta uma comparação detalhada das características técnicas de cada solução. Esta análise permitiu identificar pontos fortes e limitações de cada ferramenta, orientando o desenvolvimento da solução proposta.

\section{Proposta de Inovação e Mais-Valias}
Entre as diversas tarefas disponíveis em NLP, este trabalho concentra-se especificamente nas exigências de anotação para a tarefa de \textit{chat disentanglement}, permitindo uma implementação direta e eficiente deste tipo de tarefa.

\section{Identificação de Oportunidade de Negócio}
O projeto apresenta potencial para aplicação em contextos académicos e de investigação, especialmente em grupos que trabalham com processamento de linguagem natural e \textit{Machine Learning}.

\clearpage
\begin{table}[p]
\setlength{\footnotesep}{15pt}
\begin{minipage}{\textwidth}
\centering
\footnotesize
\begin{adjustbox}{width=\textwidth,center}
\begin{tabular}{|l|p{3.5cm}|p{3.5cm}|p{3.5cm}|p{3.5cm}|}
\hline
\textbf{Solução} & \textbf{BRAT\protect\footnotemark[1]} & \textbf{Doccano\protect\footnotemark[2]} & \textbf{Excel} & \textbf{Solução Proposta} \\
\hline
\textit{Frontend} & jQuery 1.7.1 + jQuery UI, SVG para visualização, JavaScript vanilla & Nuxt.js \textit{framework}, Vue.js, Modern JavaScript/TypeScript & Interface de folha de cálculo tradicional & React 18, TypeScript, Componentes funcionais, \textit{Hooks} customizados \\
\hline
\textit{Backend} & Python 2.5+, CGI/FastCGI, Bibliotecas JSON & Python 3.8+, Django 4.0+, REST API architecture & N/A (aplicação \textit{desktop}) & Python 3.x, FastAPI, Arquitetura REST API \\
\hline
Formato de Anotação & Ficheiros .ann para anotações, Ficheiros .txt para texto & REST API \textit{endpoints}, JSON para comunicação & Células e fórmulas manuais & REST API para comunicação; anotações via base de dados; suporte a importação/exportação CSV e JSON \\
\hline
Dados & Sistema de ficheiros, Estrutura em diretórios, Sem base de dados & Base de dados (Django ORM), Suporte a PostgreSQL & Ficheiros .xlsx/.csv locais & Base de dados relacional SQLite gerida via ORM; armazenamento estruturado \\
\hline
\textit{Deployment} & Apache/Lighttpd com CGI, Servidor Python autónomo & \textit{Docker containers}, \textit{Docker Compose} & Instalação local por utilizador & \textit{Docker containers}, Configuração simplificada \\
\hline
Extensibilidade & Sistema via ficheiros .conf, Arquitetura modular, \textit{Plugins} jQuery & Arquitetura Django modular, REST API extensível & Limitada a macros e fórmulas & Componentes React modulares, API REST extensível (FastAPI) \\
\hline
Controlo de Qualidade & Manual, sem métricas automáticas & Limitado, sem métricas de IAA & Manual, propenso a erros humanos & Cálculo automático de \textit{Inter-Annotator Agreement} (1-to-1) \\
\hline
Colaboração & Limitada, baseada em ficheiros & Suporte básico multi-utilizador & Partilha manual de ficheiros & Gestão de múltiplos anotadores, controlo de acesso por \textit{roles} \\
\hline
Estado de Manutenção & Última atualização ~2012, Projeto inativo & Projeto ativo, Atualizações regulares & Software comercial ativo & Em desenvolvimento ativo, \textit{Stack} moderna \\
\hline
Especialização & Genérica para anotação de texto & Genérica para múltiplas tarefas de NLP & Não especializada & Especializada em \textit{chat disentanglement} \\
\hline
\end{tabular}
\end{adjustbox}
\caption{Comparação técnica detalhada das soluções utilizadas no contexto do AISIC Lab}
\label{tab:tech-comparison}

\footnotetext[1]{\cite{brat-repo,brat-about}}
\footnotetext[2]{\cite{doccano-repo,doccano-docs}}
\end{minipage}
\end{table}

