\documentclass[11pt]{report}
\usepackage[pages=some]{background} % package for background
%\usepackage{background} % package for background
\usepackage{graphicx} % package for images
\usepackage{fontspec} % package for fonts
\usepackage[portuguese]{babel} % package for Portuguese language settings (hyphenation, etc.)
\usepackage{helvet} % package for font 'Helvetica'
\usepackage{color} % package for colors
\usepackage{soul} % package for text highlighting
\usepackage{fancyhdr} % package for header and footer formatting
\usepackage{titlesec} % package for chapter name formatting
\usepackage{hyperref} % package for hyperlinks
\usepackage{imakeidx} % package for indices
\usepackage[backend=biber,
            style=authoryear,
            sorting=nyt,
            maxcitenames=2,
            giveninits=true]{biblatex} % package for bibliography
\usepackage[toc]{glossaries} % package for glossaries
\usepackage{tocbibind} % package for toc listing
\usepackage[a4paper, top=2.5cm, bottom=2.5cm, left=3cm, right=3cm]{geometry} % sets page size and margins
\graphicspath{{/images}} % sets images folder
%\RequirePackage{ul-version}
\usepackage{titlesec}
\usepackage{ul-version}
\usepackage{todonotes}  % Para os comandos \todo
\usepackage{graphicx}   % Para incluir imagens
% \usepackage[round]{natbib}  % Remove or comment out this line
\usepackage{pdflscape}


% home background settings
\backgroundsetup{
    scale=1,
    opacity=1,
    angle=0,
    position=current page.center,
    contents={\includegraphics[width=\paperwidth,height=\paperheight]{images/home_background}}
    %pages=some
}

% hyperlinks setup
\hypersetup{
    linktoc=all,
    colorlinks = true,
    linkcolor={black!50!black},
    citecolor={black!50!black},
    urlcolor={black!80!black}
}

% chapter name formatting
\titleformat{\chapter}[hang]
{\normalfont\huge\bfseries}{\Huge\thechapter \hspace{0.5em}- }{0pt}{\Huge}
\titlespacing*{\chapter}{0pt}{0pt}{20pt}

% header and footer settings
\setlength{\headheight}{18pt} % adjusts head height for header compatibility
\addtolength{\topmargin}{-4.4pt} % adjusts top margin to compensate for change
\fancypagestyle{plain}{
    \pagestyle{fancy}
    \fancyhf{}
    \fancyhead[L]{\large{\textit{Ferramenta de Anotação}}} % sets header
    \fancyfoot[R]{\rule{\textwidth}{0.4pt}\\\thepage}
}

% indice settings
\makeindex
\renewcommand{\contentsname}{Índice} % updates name of the indice page

% bibliography settings
\addbibresource{bibliography.bib}

% glossary entries
\newacronym{lei}{LEI}{Licenciatura em Engenharia Informática}
\newacronym{tfc}{TFC}{Trabalho Final de Curso}
\makeglossaries

% lists settings
\renewcommand{\listfigurename}{Lista de Figuras}
\renewcommand{\listtablename}{Lista de Tabelas}

% font settings
\renewcommand{\familydefault}{\sfdefault} % sets default font family}
\setsansfont{Arial}


% document settings
\title{Ferramenta de Anotação}
\author{Fábio Lopes}
\date{Data}

\DeclareNameAlias{sortname}{family-given}
\DeclareFieldFormat{title}{#1}

\begin{document}
	
    %!TEX root = ./main.tex
\begin{titlepage}
    \BgThispage
    \centering
    \vspace*{7cm}
    
    {\fontsize{50}{1.15}\selectfont Ferramenta de Anotação}
    \vspace{3cm}

    {\fontsize{29}{1.15}\selectfont
        \textbf{Trabalho Final de Curso}\\
    }

    \vspace{1\baselineskip}
    {\fontsize{14}{1.15}\selectfont
        Relatório Final - 3ª Entrega
    }

    \vspace{5.5cm}
    {\fontsize{10}{16}\selectfont
        {Fábio Lopes - a22103261} \\
        {\textbf{Orientador:}} \\
        {Bruno Saraiva} \\
        {\textbf{Co-orientador:}} \\
        {Zuil Filho}

        \vspace{0.3cm}
        Trabalho Final de Curso $|$ LEI $|$ {2024/25}
    }
\end{titlepage}
    \input{copyright}
    %\input{agradecimentos}
    %!TEX root = ./main.tex
\chapter*{Resumo}
\addcontentsline{toc}{chapter}{Resumo}

Este trabalho apresenta o desenvolvimento de uma ferramenta de anotação que implementa um Módulo de Chat Disentanglement especializado, concebido para apoiar a criação de datasets anotados para investigação em conversation disentanglement. O módulo disponibiliza uma interface de utilizador onde os anotadores podem identificar e marcar diferentes threads de conversa que ocorrem simultaneamente em dados de chatrooms. O Chat Disentanglement, conforme descrito por \textcite{elsner2010disentangling}, é a tarefa de separar múltiplas conversas concorrentes num único canal de comunicação. A nossa ferramenta de anotação foca-se em fornecer uma experiência de utilizador intuitiva e fácil para minimizar erros de anotação, e será utilizada pelo AISIC Lab (Artificial Intelligence and Social Interaction and Complexity) para criar datasets anotados de conversas de chatrooms, que podem depois ser utilizados por investigadores e anotadores para estudar e desenvolver soluções automatizadas de disentanglement.

    %!TEX root = ./main.tex
\chapter*{Abstract}
\addcontentsline{toc}{chapter}{Abstract}

This work presents the development of an annotation tool that implements a specialized Chat Disentanglement Module, designed to support the creation of annotated datasets for conversation disentanglement research. The module provides a user interface where annotators can identify and mark different conversation threads occurring simultaneously in chatroom data. Chat disentanglement, as described by \textcite{elsner2010disentangling}, is the task of separating multiple concurrent conversations in a single communication channel. Our annotation tool focuses on providing an intuitive and easy user experience to minimize annotation errors, and will be used by the AISIC Lab (Artificial Intelligence and Social Interaction and Complexity) to create annotated datasets of chatroom conversations, which can then be utilized by researchers and annotators to study and develop automated disentanglement solutions.
    
    \tableofcontents
    \listoffigures
    \listoftables
    \printglossary[type=\acronymtype,title={Lista de Siglas e Acrónimos}]
    
    % Chapters
    \chapter{Introdução}
\label{cha:introducao}

Considerando os atuais desafios da área de Processamento de Linguagem Natural (PLN) e acompanhando as tendências emergentes, surgiu, no contexto do AISIC Lab (Artificial Intelligence, Social Interaction and Complexity), a necessidade de uma ferramenta especializada para a anotação de dados. Este projeto visa responder aos desafios específicos encontrados no laboratório, nomeadamente na análise de interações em ambientes de chat com múltiplos participantes, uma tarefa conhecida como \textit{chat disentanglement}.

\section{Enquadramento e Motivação}

A anotação de dados é um processo crítico no treino de modelos de \textit{Machine Learning}, sendo indispensável para transformar dados não estruturados — como o texto de uma conversa — em dados estruturados e anotados. São estes dados que servem como "verdade fundamental" (\textit{ground truth}), permitindo que os algoritmos de PLN aprendam a reconhecer padrões e a executar tarefas complexas com precisão.

Uma das tarefas mais desafiadoras neste domínio é o \textbf{\textit{chat disentanglement}}: o processo de separar um diálogo, que pode conter múltiplas conversas entrelaçadas, nos seus \textit{threads} (fios de conversa) constituintes. A motivação principal deste projeto emerge da ausência de ferramentas que sejam, simultaneamente, especializadas nesta tarefa e que integrem mecanismos de controlo de qualidade, como a análise da concordância entre anotadores (\textit{Inter-Annotator Agreement} - IAA).

Uma das motivações deste projeto emerge das necessidades específicas identificadas no AISIC LAB, onde a análise de mensagens em grupos de chat, requer ferramentas especializadas de anotação. 
Através de análises preliminares e feedback dos investigadores, identificámos três desafios fundamentais:

\begin{enumerate}
    \item \textbf{Complexidade das Tarefas de Anotação}:
    \begin{itemize}
        \item Necessidade de suporte a diferentes tipos de anotação
        \item Gestão de múltiplos anotadores e controle de qualidade
        \item Requisitos específicos para diferentes domínios de aplicação
    \end{itemize}

    \item \textbf{Limitações das Ferramentas Existentes}:
    \begin{itemize}
        \item Ferramentas estabelecidas como o BRAT\footnote{\cite{brat-repo,brat-about}} apresentam limitações tecnológicas e falta de evolução
        \item Soluções atuais como o Doccano\footnote{\cite{doccano-repo,doccano-docs}}, que utiliza Django como framework backend, impõem uma estrutura mais rígida, o que limita a capacidade de realizar adaptações específicas às necessidades do projeto.
        \item Necessidade de maior flexibilidade na modelação de dados e lógica aplicacional
        \item Dificuldade de integração com fluxos de trabalho existentes e específicos
    \end{itemize}

    \item \textbf{Necessidade de Integração Completa do Workflow}:
    \begin{itemize}
        \item Gestão integrada das fases pré-anotação, anotação e pós-anotação
        \item Necessidade de implementação de métricas e análises específicas
        \item Suporte a fluxos customizados de processamento de dados
        \item Integração com pipelines de NLP e análise de dados existentes
    \end{itemize}
\end{enumerate}

A decisão de desenvolver uma solução dedicada, em vez de adaptar ferramentas existentes, fundamenta-se na necessidade de maior controlo sobre todo o processo de anotação. Esta abordagem garante a flexibilidade para implementar workflows customizados de pré-processamento, anotação e análise posterior dos dados. 

O problema central que este trabalho se propõe a resolver é, portanto, o seguinte: **como criar um ambiente de software integrado que não apenas facilite a tarefa de anotação de \textit{chat disentanglement}, mas que também forneça aos gestores de projeto as ferramentas para gerir o processo e avaliar a qualidade das anotações produzidas?**


\section{Objetivos}

Para responder ao problema identificado, foram definidos e alcançados os seguintes objetivos específicos para o projeto:

\begin{enumerate}
    \item \textbf{Desenvolver uma Ferramenta Dedicada:} Construir uma aplicação web completa, com um \textit{frontend} reativo e um \textit{backend} robusto, especificamente desenhada para a tarefa de \textit{chat disentanglement}.
    
    \item \textbf{Implementar Funcionalidades de Gestão:} Dotar a plataforma de um painel de administração para a gestão de múltiplos projetos e utilizadores (anotadores e administradores), com controlo de acesso.
    
    \item \textbf{Automatizar o Cálculo de Métricas de Qualidade:} Implementar o cálculo automático do \textit{Inter-Annotator Agreement} (IAA) como uma funcionalidade central, permitindo a análise da consistência entre anotadores diretamente na plataforma.
    
    \item \textbf{Garantir a Interoperabilidade dos Dados:} Assegurar que os dados (mensagens e anotações) possam ser facilmente importados e exportados em formatos padrão (CSV e JSON), facilitando a integração da ferramenta em fluxos de trabalho de investigação mais vastos.
\end{enumerate}

A principal contribuição deste trabalho é a entrega de uma ferramenta de anotação \textit{open-source}, funcional e especializada, que preenche a lacuna identificada e suporta um fluxo de trabalho de anotação completo e integrado.

\section{Estrutura do Documento}

Este documento está organizado da seguinte forma:

\begin{itemize}
    \item \textbf{Capítulo 2 - Pertinência e Viabilidade}: Apresenta uma análise do estado da arte e compara a solução proposta com ferramentas existentes, justificando a sua relevância.

    \item \textbf{Capítulo 3 - Especificação e Modelação}: Detalha os requisitos funcionais e não-funcionais, os casos de uso e a modelação da arquitetura e da base de dados.

    \item \textbf{Capítulo 4 - Solução Proposta}: Descreve em detalhe a arquitetura final da aplicação, as tecnologias utilizadas e os seus principais componentes.

    \item \textbf{Capítulo 5 - Testes e Validação}: Apresenta a estratégia de verificação técnica utilizada para garantir a robustez e o correto funcionamento do sistema.

    \item \textbf{Capítulo 6 - Método e Planeamento}: Detalha a metodologia de desenvolvimento e realiza uma análise crítica do planeamento face à execução real do projeto.

    \item \textbf{Capítulo 7 - Resultados}: Apresenta os resultados concretos da implementação, incluindo as funcionalidades da plataforma, o fluxo de trabalho suportado e a documentação técnica gerada.

    \item \textbf{Capítulo 8 - Conclusão}: Sintetiza as contribuições do trabalho, discute as suas limitações e aponta direções para trabalhos futuros.
\end{itemize}

    \chapter{Pertinência e Viabilidade}

\section{Pertinência}
O desenvolvimento de uma ferramenta modular para anotação de dados surge como resposta a uma necessidade crítica no campo do processamento de linguagem natural (NLP). A pertinência desta solução é evidenciada por múltiplos fatores.

O AISIC LAB, validou a necessidade desta ferramenta através de feedback direto dos anotadores sobre as limitações das ferramentas atuais, bem como através da avaliação dos investigadores sobre o impacto na qualidade dos dados e análise das necessidades específicas de projetos em andamento.

O desenvolvimento desta ferramenta promete uma redução significativa no tempo de anotação, além de proporcionar uma melhoria substancial na qualidade e consistência dos dados anotados. A solução também facilita a colaboração entre anotadores e oferece suporte flexível a múltiplos tipos de tarefas de anotação, adaptando-se às necessidades específicas de cada projeto.

\section{Viabilidade}
A implementação da solução demonstra-se viável em múltiplas dimensões. Do ponto de vista técnico, a experiência prévia com um protótipo funcional em React, aliada à atual implementação utilizando FastAPI e SQLite, combinada com a disponibilidade de frameworks modernos para desenvolvimento, fornece uma base sólida para o projeto. A arquitetura modular planeada permite um desenvolvimento incremental e sustentável, aproveitando a infraestrutura existente para deployment.

Quanto à viabilidade económica, o projeto beneficia de um baixo custo de desenvolvimento inicial, principalmente devido à utilização de tecnologias open-source. Além disso, apresenta potencial para comercialização futura e promete uma redução significativa nos custos operacionais relacionados à anotação de dados.

A aceitação social do projeto é confirmada pelo feedback positivo dos utilizadores finais e pelo forte alinhamento com as necessidades do departamento. O potencial de aplicação em outros contextos académicos e a contribuição significativa para a pesquisa em NLP reforçam sua relevância no ambiente académico e científico.

\section{Análise Comparativa com Soluções Existentes}

\subsection{Soluções Existentes}

\subsubsection{Doccano}
O Doccano é uma plataforma open-source para anotação de dados em NLP. A ferramenta oferece suporte a múltiplos tipos de anotação e possui uma arquitetura modular que permite extensões dentro da sua estrutura.

\subsubsection{BRAT}
O BRAT é uma ferramenta estabelecida para anotação de texto, com foco em simplicidade e interface intuitiva. Apesar de sua maturidade, apresenta limitações em termos de extensibilidade e modernização.

\subsection{Análise de Benchmarking}

A Tabela~\ref{tab:tech-comparison} apresenta uma comparação detalhada das características técnicas de cada solução. Esta análise permitiu identificar pontos fortes e limitações de cada ferramenta, orientando o desenvolvimento da solução proposta.

\section{Proposta de Inovação e Mais-Valias}
A solução desenvolvida foca-se especificamente nas necessidades de anotação para a tarefa de chat disentanglement nesta fase inicial, permitindo uma implementação direta e eficiente deste tipo de tarefa. A abordagem modular facilita a adaptação para diferentes necessidades de anotação, com o objetivo futuro de suportar múltiplos tipos de tarefas de anotação, mantendo a simplicidade de uso.

\section{Identificação de Oportunidade de Negócio}
O projeto apresenta potencial para aplicação em contextos acadêmicos e de pesquisa, especialmente em grupos que trabalham com processamento de linguagem natural. A capacidade de adaptação para diferentes tipos de anotação permite atender necessidades específicas de diversos projetos de pesquisa.

\clearpage
\begin{table}[p]
\setlength{\footnotesep}{15pt}
\begin{minipage}{\textwidth}
\centering
\begin{tabular}{|l|p{0.25\textwidth}|p{0.25\textwidth}|p{0.25\textwidth}|}
\hline
\textbf{Solução} & \textbf{BRAT\protect\footnotemark[1]} & \textbf{Doccano\protect\footnotemark[2]} & \textbf{Solução Proposta} \\
\hline
Frontend & jQuery 1.7.1 + jQuery UI, SVG para visualização, JavaScript vanilla, XHTML templates & Nuxt.js framework, Vue.js (implícito via Nuxt), Modern JavaScript/TypeScript & React 18, TypeScript, Componentes funcionais, Hooks customizados \\
\hline
Backend & Python 2.5+, CGI/FastCGI, Bibliotecas JSON & Python 3.8+, Django 4.0+, REST API architecture & Python 3.x, FastAPI, Arquitectura REST API \\
\hline
Formato de Anotação & Arquivos .ann para anotações, Arquivos .txt para texto, JSON para comunicação & REST API endpoints, JSON para comunicação & REST API para comunicação; anotações via base de dados; suporte atual a importação CSV (planeado para mais formatos) \\
\hline
Dados & Sistema de arquivos, Estrutura em diretórios, Sem banco de dados & Database (Django ORM), Suporte a PostgreSQL & Base de dados relacional SQLite gerida via ORM; armazenamento estruturado de dados e anotações. \\
\hline
Deployment & Apache/Lighttpd com CGI, Standalone Python server & Docker containers, Docker Compose support, Production/Development configs & Docker containers, Setup simplificado \\
\hline
Extensibilidade & Sistema via arquivos .conf, Arquitetura modular, Plugins jQuery & Modular Django architecture, REST API extensibility, Frontend component system & Componentes React modulares, API REST extensível (FastAPI) \\
\hline
Estado de Manutenção & Última atualização ~2012, Tecnologias desatualizadas, Projeto inativo & Projeto ativo, Tecnologias modernas, Atualizações regulares & Em desenvolvimento ativo, Stack moderna, Iterações frequentes \\
\hline
Documentação & README detalhado, Exemplos incluídos, Tutoriais no código & Documentação estruturada, Guias de desenvolvimento, Diagramas de arquitetura & Documentação focada, Exemplos práticos, Guias de desenvolvimento \\
\hline
Requisitos Sistema & Python 2.5+, Servidor Web com CGI, Browser com SVG & Python 3.8+, Docker (recomendado), Node.js para desenvolvimento & Node.js 18+, Python 3.x, Navegador moderno, Docker (opcional) \\
\hline
Segurança & Autenticação básica, Controle via arquivo & Django authentication, Modern security practices & Autenticação baseada em roles (administrador/anotador) \\
\hline
\end{tabular}
\caption{Comparação técnica detalhada das soluções}
\label{tab:tech-comparison}

\footnotetext[1]{\cite{brat-repo,brat-about}}
\footnotetext[2]{\cite{doccano-repo,doccano-docs}}
\end{minipage}
\end{table}


    \chapter{Especificação e Modelação}
\label{cha:especificacao_modelacao}

Neste capítulo, são detalhadas as especificações técnicas da solução desenvolvida. A análise de requisitos é apresentada em primeiro lugar, seguida pela modelação da arquitetura, da base de dados e da API, que em conjunto definem a estrutura e o comportamento do sistema.

\section{Análise de Requisitos}

A tabela seguinte resume os requisitos funcionais (RF) e não-funcionais (RNF) identificados para o projeto, juntamente com o estado final da sua implementação.

% Tabela de Requisitos atualizada
\begin{table}[h!]
    \centering
    \begin{tabular}{|p{0.1\textwidth}|p{0.5\textwidth}|p{0.15\textwidth}|p{0.15\textwidth}|}
        \hline
        \textbf{ID} & \textbf{Descrição} & \textbf{Estado} & \textbf{Notas} \\
        \hline
        \multicolumn{4}{|c|}{\textbf{Requisitos Funcionais}} \\
        \hline
        RF1 & Suporte à importação de dados (CSV, JSON) & Cumprido & Implementada importação de mensagens via CSV e anotações via JSON. \\
        \hline
        RF2 & Organização de dados por projetos & Cumprido & Estrutura central da aplicação. \\
        \hline
        RF3 & Exportação dos dados anotados (JSON) & Cumprido & Funcionalidade de exportação por sala de chat implementada. \\
        \hline
        RF4 & Interface de utilizador clara e funcional & Cumprido & Interface React desenvolvida com foco na usabilidade para anotação. \\
        \hline
        RF5 & Suporte a múltiplos anotadores por projeto & Cumprido & Sistema de atribuição de utilizadores a projetos. \\
        \hline
        RF6 & Gestão de atribuição de tarefas & Cumprido & Administradores podem atribuir/remover utilizadores de projetos. \\
        \hline
        RF7 & Interface especializada para chat disentanglement & Cumprido & Ecrã de anotação dedicado com gestão de threads. \\
        \hline
        RF8 & Sistema de tagging para classificação em threads & Cumprido & Funcionalidade nuclear da anotação. \\
        \hline
        RF9 & Visualização sequencial das mensagens & Cumprido & A sala de chat apresenta as mensagens por ordem. \\
        \hline
        RF10 & Armazenamento para cálculo de métricas & Cumprido & O modelo de dados permite o cálculo de IAA. \\
        \hline
        RF11 & Autenticação de utilizadores & Cumprido & Implementado com JWT (access e refresh tokens). \\
        \hline
        RF12 & Definição de roles (admin/anotador) & Cumprido & O modelo `User` contém o campo `is\_admin`. \\
        \hline
        RF13 & Controlo de acesso baseado em permissões & Cumprido & Endpoints da API protegidos com base no role. \\
        \hline
        \multicolumn{4}{|c|}{\textbf{Requisitos Não Funcionais}} \\
        \hline
        RNF1 & Tempo de resposta adequado & Cumprido & A API responde rapidamente a pedidos interactivos. \\
        \hline
        RNF2 & Processamento eficiente para múltiplos utilizadores & Parcialmente & A arquitetura suporta-o, mas não foram feitos testes de carga formais. \\
        \hline
        RNF3 & Interface responsiva & Parcialmente & O design é funcional mas não foi totalmente otimizado para todos os tamanhos de ecrã. \\
        \hline
        RNF4 & Feedback visual claro & Cumprido & A interface React fornece feedback sobre o estado (loading, error, success). \\
        \hline
        RNF5 & Interface simples e intuitiva & Cumprido & O design focou-se na simplicidade para a tarefa de anotação. \\
        \hline
        RNF6 & Backup automático de anotações & Não Cumprido & Não implementado. Requer uma solução de infraestrutura (e.g., cron jobs na base de dados). \\
        \hline
        RNF7 & Logging de atividades críticas & Parcialmente & O backend utiliza logging básico, mas não um sistema de logging estruturado e exaustivo. \\
        \hline
    \end{tabular}
    \caption{Tabela de Requisitos e Estado de Implementação.}
    \label{tab:requisitos}
\end{table}

\section{Modelação da Solução}

A modelação da solução foi um processo iterativo que resultou numa arquitetura de três camadas, um modelo de dados relacional robusto e uma API RESTful bem definida.

\subsection{Arquitetura da Solução}

A plataforma segue uma arquitetura cliente-servidor moderna e desacoplada, composta por dois componentes principais:

\begin{itemize}
    \item \textbf{Frontend (Cliente):} Uma Single-Page Application (SPA) desenvolvida com a biblioteca \textbf{React}. É responsável exclusivamente pela apresentação da interface do utilizador e pela gestão do estado local da UI. Toda a lógica de negócio e manipulação de dados é delegada ao backend através de chamadas à API.
    \item \textbf{Backend (Servidor):} Uma API RESTful desenvolvida com a framework \textbf{FastAPI} (Python). É responsável por toda a lógica de negócio, incluindo a autenticação de utilizadores, controlo de permissões, operações na base de dados (CRUD) e cálculos de métricas como o IAA.
\end{itemize}
Esta separação estrita de responsabilidades (`separation of concerns`) garante uma maior manutenibilidade, escalabilidade e a possibilidade de desenvolver e testar os dois componentes de forma independente.

\subsection{Modelo de Dados}

O coração do backend é o seu modelo de dados relacional, implementado com \textbf{SQLAlchemy} como ORM (Object-Relational Mapper), que mapeia as classes Python para tabelas numa base de dados SQL. A Figura~\ref{fig:modelo-er} apresenta o Diagrama de Entidade-Relação (ERD) final do sistema.

As entidades centrais do modelo são:
\begin{itemize}
    \item \textbf{User, Project, e ProjectAssignment:} que em conjunto gerem os utilizadores e as suas permissões de acesso aos diferentes projetos.
    \item \textbf{ChatRoom e ChatMessage:} que estruturam os dados a serem anotados.
    \item \textbf{Annotation:} a tabela principal que armazena a ligação entre uma mensagem, um anotador e um \textit{thread}, sendo a base para toda a análise posterior.
\end{itemize}
A utilização do SQLAlchemy e do Alembic para migrações de base de dados permitiu uma evolução estruturada do esquema ao longo do desenvolvimento do projeto.

\subsection{API RESTful}

A comunicação entre o frontend e o backend é realizada através de uma API RESTful. A API expõe um conjunto de endpoints para todas as operações necessárias, desde a autenticação até ao cálculo de métricas. A API está documentada seguindo o standard OpenAPI, o que facilita a sua exploração e integração.

Os endpoints estão logicamente agrupados por responsabilidade:
\begin{itemize}
    \item \textbf{Auth:} Gestão de autenticação e tokens.
    \item \textbf{Projects:} Operações de utilizador normal sobre projetos e salas de chat.
    \item \textbf{Annotations:} Criação e gestão de anotações.
    \item \textbf{Admin:} Operações de administração, incluindo importação/exportação e cálculo de IAA.
\end{itemize}

\section{Protótipos de Interface}

\subsection{Mapa de Navegação}

A estrutura de navegação da plataforma está idealizada para adaptar-se aos diferentes perfis de utilizador. O sistema prevê um portal de login onde após autenticação, os utilizadores terão acesso a um dashboard principal, que funcionará como ponto central de acesso às diversas funcionalidades da plataforma. A estrutura completa do mapa de navegação pode ser visualizada na Figura~\ref{fig:mapa-navegacao}, que foi modificada para ocupar uma página inteira em landscape.

A ferramenta de anotação será estruturada de forma modular, onde o dashboard principal permitirá acesso aos vários módulos disponíveis. Numa primeira fase, o módulo de disentanglement será o unico modulo desta arquitetura modular, no entanto o objetico da ferramenta de anotação será de acomodar mais módulos no futuro.

\subsection{Principais Interfaces}

\subsubsection{Portal de Entrada}
O acesso à plataforma é controlado através de um portal de autenticação minimalista. Esta decisão de design visa garantir a integridade dos dados e a atribuição correta das tarefas de anotação, sendo o login obrigatório para qualquer interação com os dados do sistema.

\subsubsection{Dashboard Principal}
Após autenticação, o utilizador acede a um dashboard que apresenta uma visão geral da plataforma. Este componente central adapta-se dinamicamente ao perfil do utilizador (administrador ou anotador), apresentando as funcionalidades relevantes e o estado atual das tarefas atribuídas.

\subsubsection{Módulo de Disentanglement}
O primeiro módulo implementado na plataforma foca-se na tarefa de disentanglement de chat, apresentando duas visões distintas:

\paragraph{Visão do Anotador}
Para os anotadores, a interface apresenta:
\begin{itemize}
    \item Lista das chatrooms atribuídas automaticamente pelo sistema
    \item Interface de anotação com visualização sequencial das mensagens
    \item Sistema intuitivo de tagging para classificação de threads
    \item Indicadores de progresso da tarefa
    \item Mecanismos de validação em tempo real
\end{itemize}

\paragraph{Visão do Administrador}
Os administradores têm acesso a funcionalidades adicionais:
\begin{itemize}
    \item Gestão completa dos datasets
    \item Monitorização do progresso dos anotadores
    \item Visualização de métricas e estatísticas
    \item Configuração da distribuição automática de tarefas
\end{itemize}

\subsubsection{Interface de Anotação}
O componente central do módulo de disentanglement é a interface de anotação, que foi projetada para maximizar a eficiência do processo de anotação. Cada chatroom é apresentada como uma sequência temporal de mensagens, onde o anotador pode facilmente:
\begin{itemize}
    \item Visualizar o contexto completo da conversa
    \item Criar e atribuir tags de thread às mensagens
    \item Acompanhar o progresso da anotação em tempo real
    \item Navegar eficientemente entre diferentes chatrooms
\end{itemize}

O sistema mantém um salvamento automático do progresso, permitindo que os anotadores retomem seu trabalho de forma seamless em qualquer momento.


\begin{figure}[p]
    \centering
    \includegraphics[width=0.95\textwidth,height=0.95\textheight,keepaspectratio]{images/2A-ERT-B.drawio.png}
    \caption{Modelo de Entidade-Relação final do Sistema.}
    \label{fig:modelo-er}
\end{figure}

\begin{landscape}
    \begin{figure}[p]
        \centering
        \makebox[\textwidth][c]{%
            \includegraphics[width=0.8\paperheight, angle=0, keepaspectratio]{images/mapaDeNavegacaoSistema.drawio.png}
        }
        \caption{Mapa de Navegação do Sistema.}
        \label{fig:mapa-navegacao}
    \end{figure}
\end{landscape}
    \chapter{Solução Proposta}

\section{Apresentação}

A solução materializa-se numa plataforma web para anotação de dados de chat, com foco inicial no processo de disentanglement. O desenvolvimento baseou-se num protótipo funcional \cite{prototype} que permitiu validar conceitos base e estabelecer fundações para evolução futura.

\paragraph{Protótipo Inicial}
O protótipo, desenvolvido entre outubro e novembro de 2024, implementou funcionalidades essenciais:

\begin{itemize}
    \item Interface base de visualização de chat
    \item Sistema de gestão de tags para anotação
    \item Processamento de ficheiros CSV com formato específico
    \item Gestão básica de workspace para ficheiros
\end{itemize}

\paragraph{Requisitos de Dados}
O sistema processa ficheiros CSV com estrutura específica, incluindo:

\begin{itemize}
    \item user\_id: identificador do utilizador
    \item turn\_id: identificador único da mensagem
    \item turn\_text: conteúdo da mensagem
    \item reply\_to\_turn: referência explícita a outra mensagem
    %\item thread: identificação da thread (preenchido pelo anotador)
\end{itemize}

\paragraph{Insights do Protótipo}
O desenvolvimento exploratório inicial proporcionou descobertas importantes:

\begin{itemize}
    \item Validação da interface de anotação para disentanglement
    \item Confirmação da viabilidade do processamento de ficheiros CSV
    \item Identificação de pontos de optimização no fluxo de trabalho
    \item Feedback directo dos utilizadores sobre funcionalidades essenciais
\end{itemize}

\section{Arquitectura}

A arquitectura da solução, representada na Figura~\ref{fig:diagrama-arquitectura}, incorpora elementos de diferentes abordagens analisadas no benchmarking. Esta decisão reflecte-se em dois princípios base:

\paragraph{Operacionalização}
O primeiro foca-se na operacionalização através de:

\begin{itemize}
    \item Deployment via containers para simplicidade operacional
    \item Configuração reduzida para facilitar manutenção
    \item Monitorização integrada de componentes
    \item Backup e recuperação de dados simplificados
\end{itemize}

\paragraph{Modularidade}
O segundo centra-se na modularidade através de:

\begin{itemize}
    \item Interfaces bem definidas entre componentes
    \item Separação clara de responsabilidades
    \item Sistema de plugins para extensões
    \item Gestão independente de dados por módulo
\end{itemize}

\paragraph{Evolução Planeada}
A arquitectura suporta evolução em várias dimensões:

\begin{itemize}
    \item Adição de novos módulos de anotação
    \item Integração com ferramentas de análise
    \item Expansão das capacidades de processamento
    \item Adaptação a diferentes tipos de dados
\end{itemize}

\section{Tecnologias e Ferramentas Utilizadas}

A implementação assenta em tecnologias web estabelecidas, escolhidas conforme estado da arte e conforme os requisitos do projeto. O frontend utiliza React para a interface de anotação, enquanto o backend em Python gere o processamento de dados, permitindo integração eficiente com ferramentas de análise e processamento.

\paragraph{Frontend}
A escolha do React como framework principal mantém-se desde o protótipo inicial, fundamentada por:

\begin{itemize}
    \item Gestão eficiente de estado através de Hooks
    \item Componentes reutilizáveis para consistência da interface
    \item Rendering optimizado para operações frequentes
    \item Extensa documentação e comunidade activa
\end{itemize}

\paragraph{Backend}
O desenvolvimento em Python permite:

\begin{itemize}
    \item Integração com bibliotecas de processamento de dados
    \item Implementação eficiente de métricas e análises
    \item Alinhamento com o ecossistema do departamento
    \item Extensibilidade para funcionalidades futuras
\end{itemize}

\section{Ambientes de Teste e de Produção}

O desenvolvimento segue uma abordagem iterativa com dois ambientes distintos:

\paragraph{Ambiente de Desenvolvimento}
Suporta o desenvolvimento activo e testes:

\begin{itemize}
    \item Configuração simplificada para desenvolvimento local
    \item Dados de teste para validação de funcionalidades
    \item Ferramentas de debugging e monitorização
    \item Automatização de testes unitários e de integração
\end{itemize}

\paragraph{Ambiente de Produção}
Garante estabilidade e performance:

\begin{itemize}
    \item Configuração optimizada para performance
    \item Backup automático de dados
    \item Monitorização de métricas operacionais
    \item Gestão de logs e diagnóstico
\end{itemize}

\section{Abrangência}

A solução abrange inicialmente o processo de disentanglement de conversas em chatrooms, estabelecendo bases para expansão futura. A arquitectura modular permite adicionar novos tipos de anotação sem alterações estruturais significativas.

\paragraph{Módulo de Disentanglement}
O primeiro módulo implementado foca-se em:

\begin{itemize}
    \item Interface especializada para visualização de chatrooms
    \item Ferramentas para identificação e separação de conversas
    \item Sistema de validação de anotações
    \item Métricas de progresso e qualidade
\end{itemize}

\paragraph{Extensibilidade}
A arquitectura suporta expansão através de:

\begin{itemize}
    \item Interfaces bem definidas para novos módulos
    \item Gestão independente de dados por módulo
    \item Documentação para desenvolvimento de extensões
\end{itemize}

\section{Componentes}

\subsection{Frontend}

O frontend da aplicação assenta em React 18, escolha fundamentada pela experiência bem-sucedida do protótipo inicial e pelas capacidades da framework para desenvolvimento de interfaces complexas. A implementação aproveita características modernas do React, como Hooks e componentes funcionais, permitindo uma gestão de estado eficiente e código mais manutenível.

A arquitectura do frontend organiza-se em componentes modulares, cada um com responsabilidades bem definidas:

\paragraph{Sistema de Componentes}
O desenvolvimento segue uma abordagem baseada em componentes reutilizáveis, organizados em três níveis:

\begin{itemize}
    \item \textbf{Componentes Base}: Elementos UI fundamentais como botões, inputs e cards, implementados com styled-components para consistência visual
    \item \textbf{Componentes Compostos}: Agregações de componentes base que implementam funcionalidades específicas, como o visualizador de mensagens ou o sistema de tagging
    \item \textbf{Páginas}: Composições completas que integram múltiplos componentes para criar interfaces funcionais
\end{itemize}

\paragraph{Gestão de Estado}
A gestão de estado utiliza uma combinação de React Hooks nativos e contextos, evitando a complexidade adicional de soluções como Redux. Esta decisão baseou-se na experiência do protótipo, onde se verificou que:

\begin{itemize}
    \item O estado local com useState é suficiente para a maioria dos componentes
    \item useContext permite compartilhar estado eficientemente entre componentes relacionados
    \item useReducer oferece gestão de estado mais complexa quando necessário
\end{itemize}

\paragraph{Interface de Anotação}
O componente central da aplicação - a interface de anotação - foi redesenhado com base no feedback do protótipo. Implementa:

\begin{itemize}
    \item Visualização em split-view das mensagens e threads identificadas
    \item Sistema de drag-and-drop para classificação de mensagens
    \item Preview em tempo real das alterações
    \item Atalhos de teclado para operações frequentes
\end{itemize}

\subsection{Backend}

O backend da aplicação será desenvolvido em Python, escolha fundamentada pela necessidade de integração com ferramentas de processamento de dados e análise. A implementação utilizará uma web framework Python moderna (Flask ou FastAPI), permitindo o desenvolvimento de uma API REST robusta e eficiente.

\paragraph{Arquitectura do Backend}
O backend segue uma arquitectura em camadas:

\begin{itemize}
    \item \textbf{API Layer}: Implementação de endpoints REST utilizando uma web framework Python (Flask ou FastAPI)
    \item \textbf{Service Layer}: Lógica de negócio e processamento de dados
    \item \textbf{Data Layer}: Gestão de persistência e acesso a dados
\end{itemize}

\paragraph{Processamento de Dados}
O sistema implementa um pipeline de processamento específico para chatrooms:

\begin{itemize}
    \item Parsing e validação de ficheiros CSV
    \item Estruturação de conversas em formato adequado para anotação
    \item Cálculo de métricas e estatísticas
    \item Cache de resultados frequentes
\end{itemize}

\paragraph{Sistema de Persistência}
A persistência de dados será implementada utilizando uma base de dados SQL leve como SQLite, através de um ORM para Python. Esta abordagem permite:

\begin{itemize}
    \item Gestão eficiente de dados estruturados
    \item Queries optimizadas para diferentes volumes de dados
    \item Backup e versionamento de dados
\end{itemize}

\subsection{Fluxo de Dados}

O fluxo de dados na plataforma foi desenhado para reflectir as necessidades práticas do processo de anotação:

\paragraph{Importação}
Os administradores podem carregar ficheiros CSV contendo conversas de chatrooms. O processo é deliberadamente simplificado, permitindo:

\begin{itemize}
    \item Upload directo de ficheiros
    \item Validação automática do formato
    \item Feedback imediato sobre a qualidade dos dados
\end{itemize}

\paragraph{Processamento}
O backend processa os ficheiros, preparando-os para anotação:

\begin{itemize}
    \item Validação estrutural dos dados
    \item Organização das mensagens em sequência temporal
    \item Preparação para disentanglement
\end{itemize}

\paragraph{Distribuição}
As conversas são disponibilizadas através da interface web:

\begin{itemize}
    \item Atribuição automática aos anotadores
    \item Tracking de progresso
    \item Sistema de validação em tempo real
\end{itemize}

\section{Interfaces}

A comunicação entre componentes realiza-se através de uma API REST, permitindo independência entre frontend e backend. Esta abordagem facilita:

\begin{itemize}
    \item Desenvolvimento paralelo de componentes
    \item Manutenção e evolução independente
    \item Integração com ferramentas externas
    \item Testes isolados de funcionalidades
\end{itemize}

O sistema de dados organiza-se por módulos, onde cada tipo de anotação mantém seu próprio modelo de dados. Para o disentanglement, isto inclui:

\begin{itemize}
    \item Estrutura de dados optimizada para chatrooms
    \item Sistema de versionamento de anotações
    \item Métricas específicas para avaliação
    \item Export em formatos standard
\end{itemize}

O diagrama apresentado na Figura~\ref{fig:diagrama-arquitectura} detalha os principais componentes da solução, incluindo o frontend em React, backend em Python, sistema de persistência de dados e as interfaces de comunicação entre os diferentes módulos. A arquitectura modular permite a extensão futura com novos tipos de anotação além do módulo inicial de disentanglement.

\begin{landscape}
    \begin{figure}[p]
        \centering
        \makebox[\textwidth][c]{%
            \includegraphics[width=0.60\paperheight, angle=0, keepaspectratio]{images/diagramaArquitetura.drawio.png}
        }
        \caption{Diagrama de Arquitectura do Sistema.}
        \label{fig:diagrama-arquitectura}
    \end{figure}
\end{landscape}
    %!TEX root = ../main.tex
\chapter{Testes e Validação}

\section{Introdução}

Este capítulo descreve como a ferramenta de anotação foi testada e validada. O objetivo é verificar se a ferramenta funciona corretamente, se é adequada para a tarefa de \textit{chat disentanglement}, e se representa uma melhoria face a métodos de anotação mais manuais. A validação pretende confirmar a aplicabilidade prática da solução desenvolvida e demonstrar o cumprimento dos objetivos propostos, nomeadamente o de contribuir para a solução de um problema real de investigação.

\section{Abordagem e Justificação dos Testes}

\subsection{Estratégia de Validação}
A validação da ferramenta baseou-se numa abordagem de avaliação técnica interna realizada em conjunto com a equipa do laboratório AISIC. Esta estratégia foi escolhida por permitir:
\begin{itemize}
    \item Uma análise técnica detalhada das funcionalidades implementadas.
    \item Avaliação por utilizadores com conhecimento especializado na área de processamento de linguagem natural.
    \item Validação em contexto real de investigação, com dados e cenários autênticos.
    \item Identificação de limitações e áreas de melhoria numa fase adequada do desenvolvimento.
\end{itemize}

\subsection{Critérios de Aceitação}
Os testes foram estruturados para validar o cumprimento dos seguintes critérios de aceitação:
\begin{itemize}
    \item \textbf{Funcionalidade Técnica:} A ferramenta deve importar, processar e exportar dados de conversas corretamente.
    \item \textbf{Adequação à Tarefa:} A interface deve facilitar a realização de tarefas de \textit{chat disentanglement}.
    \item \textbf{Cálculo de Métricas:} O sistema deve calcular automaticamente métricas de concordância inter-anotador.
    \item \textbf{Gestão de Projetos:} O sistema deve permitir gestão adequada de utilizadores e projetos de anotação.
    \item \textbf{Eficiência Operacional:} A ferramenta deve representar uma melhoria face a métodos manuais.
\end{itemize}

\subsection{Participantes e Perfis de Avaliação}
A avaliação envolveu dois perfis principais de investigadores:
\begin{itemize}
    \item \textbf{Administradores:} Investigadores familiarizados com a gestão de sistemas e projetos de investigação, que avaliaram as funcionalidades administrativas da ferramenta.
    \item \textbf{Anotadores:} Um grupo mais alargado de participantes (investigadores, estudantes) com experiência ou interesse na análise de dados textuais, que se focaram na tarefa central de anotação para \textit{chat disentanglement}.
\end{itemize}

\section{Cenários e Tarefas de Validação}

As tarefas de validação procuraram simular o ciclo de vida completo da utilização da ferramenta:

\begin{itemize}
    \item \textbf{Tarefas de Administração:} Avaliadas pelos investigadores com perfil administrativo, incluíram:
    \begin{itemize}
        \item Criação e gestão de projetos de anotação.
        \item Gestão de utilizadores (criação, atribuição de papéis e permissões).
        \item Importação de dados (ficheiros de \textit{chat}) para os projetos.
        \item Verificação da correta importação e organização dos dados.
    \end{itemize}
    \item \textbf{Tarefas de Anotação:} Avaliadas pelos investigadores com experiência em anotação, incluíram:
    \begin{itemize}
        \item Navegação e visualização das conversas importadas.
        \item Utilização da interface para identificar e marcar diferentes \textit{threads}.
        \item Atribuição de mensagens às \textit{threads} correspondentes.
        \item Edição e correção de anotações realizadas.
        \item Exportação das anotações finalizadas.
    \end{itemize}
\end{itemize}

\section{Validação Técnica e de Recursos}

\subsection{Validação da Infraestrutura Técnica}
Foi realizada uma validação detalhada dos recursos computacionais e técnicos da solução:
\begin{itemize}
    \item \textbf{Armazenamento:} Validação da capacidade de armazenamento de dados na base SQLite, incluindo testes com conjuntos de dados de diferentes dimensões.
    \item \textbf{Desempenho:} Avaliação dos tempos de resposta para operações críticas (importação, cálculo de métricas, exportação).
    \item \textbf{Estabilidade:} Testes de funcionamento contínuo e robustez da aplicação durante sessões prolongadas de anotação.
    \item \textbf{Compatibilidade:} Verificação do funcionamento em diferentes sistemas operativos e navegadores web.
\end{itemize}

\subsection{Validação da Importação e Gestão de Dados}
Foi realizada uma validação técnica detalhada dos processos de importação de dados, verificando:
\begin{itemize}
    \item Correta estruturação e organização dos dados na base de dados.
    \item Integridade dos dados durante o processo de importação.
    \item Funcionalidade de exportação dos dados anotados em formatos apropriados para análise posterior.
\end{itemize}

\subsection{Avaliação da Interface de Anotação}
A interface de anotação foi avaliada em reunião pela equipa do laboratório, focando:
\begin{itemize}
    \item Adequação da visualização das conversas para a tarefa de disentanglement.
    \item Facilidade de identificação e marcação de diferentes \textit{threads}.
    \item Eficiência do processo de atribuição de mensagens a \textit{threads}.
    \item Capacidade de edição e correção de anotações.
    \item Clareza visual da organização das anotações realizadas.
\end{itemize}

\subsection{Validação das Métricas de Concordância}
Uma funcionalidade central da ferramenta é o cálculo automático de métricas de concordância inter-anotador (IAA - \textit{Inter-Annotator Agreement}). Foi validado:
\begin{itemize}
    \item Correto funcionamento do algoritmo de cálculo de concordância "one-to-one" \cite{elsner2008you}.
    \item Adequação das métricas implementadas para a avaliação da qualidade das anotações.
    \item Capacidade de comparação entre múltiplos anotadores no mesmo conjunto de dados.
    \item Funcionalidade de exportação das métricas calculadas para análise externa.
\end{itemize}

\section{Resultados da Validação}

\subsection{Cumprimento dos Critérios de Aceitação}
A validação confirmou o cumprimento de todos os critérios de aceitação estabelecidos:

\begin{itemize}
    \item \textbf{Funcionalidade Técnica:} Os processos de importação e exportação de dados funcionam de forma robusta.
    \item \textbf{Adequação à Tarefa:} A interface de anotação permite realizar eficientemente as tarefas de \textit{chat disentanglement}.
    \item \textbf{Cálculo de Métricas:} As métricas de concordância são calculadas automaticamente e fornecem informação útil sobre a qualidade das anotações.
    \item \textbf{Gestão de Projetos:} O sistema de gestão de utilizadores e projetos adequa-se às necessidades de um ambiente de investigação.
    \item \textbf{Eficiência Operacional:} Confirmada melhoria significativa face a métodos manuais.
\end{itemize}

\subsection{Validação de Recursos Técnicos}
Os testes de infraestrutura confirmaram:
\begin{itemize}
    \item Desempenho adequado para conjuntos de dados típicos de investigação.
    \item Compatibilidade com os principais navegadores web e sistemas operativos.
    \item Capacidade de armazenamento suficiente para múltiplos projetos simultâneos.
\end{itemize}

\subsection{Adequação ao Fluxo de Trabalho}
A avaliação em reunião pela equipa do laboratório indicou que as tarefas de anotação foram avaliadas como mais eficientes quando realizadas com a ferramenta, comparativamente a abordagens manuais ou anotações realizadas em outras ferramentas citadas neste trabalho. Especificamente:
\begin{itemize}
    \item A visualização estruturada das conversas facilita a identificação de \textit{threads}.
    \item O processo de atribuição de mensagens é mais intuitivo que métodos totalmente manuais.
    \item O cálculo automático de métricas de concordância elimina a necessidade de processamento manual posterior.
    \item A capacidade de exportação permite integração com fluxos de trabalho de análise existentes.
\end{itemize}

\subsection{Limitações Identificadas}
Durante a validação, foram também identificadas algumas limitações:
\begin{itemize}
    \item A ferramenta é específica para tarefas de \textit{chat disentanglement}, não sendo diretamente aplicável a outros tipos de anotação.
    \item A interface assume familiaridade básica com conceitos de anotação de dados textuais.
\end{itemize}

\section{Conclusões da Validação}

A validação realizada confirma que a ferramenta desenvolvida cumpre os objetivos propostos, fornecendo uma solução técnica adequada para tarefas de \textit{chat disentanglement}. Todos os critérios de aceitação foram cumpridos, e a avaliação pela equipa do laboratório indica que a ferramenta representa uma melhoria significativa face a métodos manuais.

A validação técnica de recursos confirmou a adequação da infraestrutura para o contexto de investigação, com desempenho e estabilidade apropriados para as necessidades identificadas. A automatização do cálculo de métricas de concordância e a estruturação do processo de anotação constituem as principais contribuições para a melhoria do fluxo de trabalho.

Embora reconhecendo as limitações inerentes a uma validação técnica interna e a especificidade da ferramenta para \textit{chat disentanglement}, os resultados fornecem indicações sobre a pertinência e eficácia da solução desenvolvida, confirmando a sua adequação ao contexto de investigação em processamento de linguagem natural e análise de conversas.

    \chapter{Método e Planeamento}

\section{Metodologia de Desenvolvimento}

O desenvolvimento deste projeto segue uma abordagem iterativa e incremental, alinhada com metodologias adaptadas ao contexto académico e às necessidades específicas do AISIC LAB. Esta escolha fundamenta-se na necessidade de validação contínua com stakeholders e na natureza evolutiva dos requisitos de uma plataforma modular de anotação.

\subsection{Princípios Metodológicos}

A metodologia adotada assenta em três princípios fundamentais:

\begin{itemize}
    \item \textbf{Iterações Curtas}: Ciclos de desenvolvimento de duas semanas, permitindo feedback regular e ajustes frequentes
    \item \textbf{Validação Contínua}: Envolvimento regular dos stakeholders do AISIC LAB para validação de funcionalidades
    \item \textbf{Desenvolvimento Incremental}: Construção progressiva da plataforma, começando pelo módulo de disentanglement
\end{itemize}

\subsection{Organização do Trabalho}

O desenvolvimento está estruturado em sprints quinzenais, com os seguintes elementos:

\begin{itemize}
    \item \textbf{Planeamento}: Definição de objetivos e tarefas no início de cada sprint
    \item \textbf{Desenvolvimento}: Implementação das funcionalidades priorizadas
    \item \textbf{Revisão}: Avaliação do progresso e demonstração aos stakeholders
    \item \textbf{Retrospetiva}: Análise do processo e identificação de melhorias
\end{itemize}

\section{Planeamento e Cronograma}

O planeamento do projeto, representado na Figura~\ref{fig:gantt-chart}, está organizado em fases distintas que refletem a evolução da plataforma desde o protótipo inicial até à solução final.

\subsection{Fases do Projeto}

\subsubsection{Fase Inicial (Outubro - Dezembro 2024)}
Esta fase focou-se na validação de conceitos e estabelecimento de fundações:
\begin{itemize}
    \item Desenvolvimento do protótipo
    \item Validação da interface de anotação
    \item Levantamento tecnológico
    \item Documentação inicial
\end{itemize}

\subsubsection{MVP - Módulo Disentanglement (Dezembro 2024 - Janeiro 2025)}
Consolidação do protótipo existente:
\begin{itemize}
    \item Refinamento da interface frontend
    \item Testes de usabilidade
    \item Implementação de feedback inicial
    \item Validação com utilizadores piloto
\end{itemize}

\subsubsection{Infraestrutura Base (Janeiro - Março 2025)}
Estabelecimento da arquitetura modular:
\begin{itemize}
    \item Setup do ambiente de desenvolvimento
    \item Migração do backend para Python
    \item Implementação da arquitetura modular
    \item Sistema de autenticação
\end{itemize}

\subsubsection{Plataforma Core (Março - Maio 2025)}
Desenvolvimento das funcionalidades principais:
\begin{itemize}
    \item Framework para múltiplos módulos
    \item Sistema de gestão de datasets
    \item API base para integração
    \item Interface de administração
\end{itemize}

\subsubsection{Finalização (Maio - Junho 2025)}
Preparação para disponibilização:
\begin{itemize}
    \item Testes extensivos
    \item Validação com utilizadores
    \item Documentação técnica
    \item Deployment em produção
\end{itemize}

\section{Análise Crítica ao Planeamento}

A presente secção visa analisar o progresso do projeto face ao planeamento inicial, identificando os principais marcos alcançados, os desafios encontrados e as adaptações realizadas durante o desenvolvimento subsequente à primeira entrega.

\subsection{Progresso Realizado}
Desde a validação inicial do conceito através do protótipo, o desenvolvimento focou-se na construção da infraestrutura base e do módulo principal de \textit{chat disentanglement}. Os principais avanços incluem:
\begin{itemize}
    \item Implementação do \textit{backend} utilizando a framework FastAPI em Python, estabelecendo uma API REST para comunicação com o \textit{frontend}.
    \item Configuração da persistência de dados com uma base de dados SQLite, gerida através do ORM SQLAlchemy.
    \item Desenvolvimento e integração do sistema de autenticação e gestão de utilizadores com papéis (administrador/anotador).
    \item Refinamento e desenvolvimento contínuo do \textit{frontend} em React, implementando as interfaces necessárias para a gestão de projetos e a anotação de \textit{disentanglement}.
    \item Implementação da funcionalidade de importação de dados a partir de ficheiros CSV.
\end{itemize}
Este progresso permitiu obter uma versão funcional da ferramenta centrada na sua tarefa principal, conforme pode ser visualizado no cronograma atualizado (Figura~\ref{fig:gantt-chart}).

\subsection{Desafios Encontrados e Adaptações ao Plano}
Durante o desenvolvimento, alguns desafios e constrangimentos levaram a ajustes no plano e na abordagem inicial:

\begin{itemize}
    \item \textbf{Priorização vs. Generalidade:} A visão inicial de uma plataforma altamente modular e genérica, capaz de suportar diversos tipos de anotação de forma extensível, revelou-se demasiado ambiciosa para o âmbito temporal e os recursos disponíveis no TFC. Para garantir a entrega de uma solução funcional e útil para o caso de uso principal do AISIC Lab, foi tomada a decisão consciente de **priorizar o desenvolvimento do módulo de \textit{chat disentanglement}**. Isto implicou que a implementação do \textit{backend} e do modelo de dados fosse mais específica para esta tarefa, adiando a introdução das abstrações necessárias para uma generalização mais ampla a outros tipos de anotação.
    \item \textbf{Foco no Formato CSV:} Pelas mesmas razões de priorização e alinhamento com as necessidades imediatas do caso de uso, o suporte à importação de dados foi, nesta fase, limitado ao formato CSV. O suporte a outros formatos (JSON, TXT, etc.) permanece como um objetivo para evolução futura.
    \item \textbf{Complexidade Técnica:} A integração entre o frontend e o backend, a gestão do estado da aplicação e a implementação correta das interações de anotação apresentaram desafios técnicos que exigiram tempo e esforço de desenvolvimento significativo.
    \item \textbf{Balanceamento de Funcionalidades:} Foi necessário balancear o desenvolvimento de novas funcionalidades com a necessidade de refinar e estabilizar as funcionalidades existentes, respondendo também a feedback e requisitos que emergiram durante o processo.
\end{itemize}
Estas adaptações, embora desviando parcialmente da visão mais abrangente inicial, foram consideradas necessárias para garantir a viabilidade da entrega de um produto funcional e relevante dentro do contexto do TFC. A modularidade e a extensibilidade continuam a ser princípios orientadores para o futuro da plataforma.

\subsection{Implicações no Planeamento Futuro}
As decisões de priorização refletem-se no estado atual da ferramenta e no planeamento subsequente. O foco continuará a ser a consolidação e teste do módulo de \textit{disentanglement}. A introdução de maior generalidade no backend e o suporte a outros formatos de dados são agora considerados trabalhos futuros, a serem abordados após a validação e estabilização da funcionalidade principal. A documentação técnica e os testes de usabilidade (Capítulo 5) ganham maior ênfase para garantir a qualidade da entrega atual.

\begin{landscape}
    \begin{figure}[p]
        \centering
        \makebox[\textwidth][c]{%
            \includegraphics[width=0.98666\paperheight, angle=0, keepaspectratio]{images/Gant.drawio.png}
        }
        \caption{Cronograma detalhado do projeto (Gantt Chart) - Estado Atualizado}
        \label{fig:gantt-chart}
    \end{figure}
\end{landscape}

    \chapter{Resultados}
\label{cha:resultados}

Este capítulo apresenta os resultados concretos obtidos com o desenvolvimento da ferramenta de anotação. A apresentação está dividida em três áreas principais: uma visão geral da plataforma final, uma análise detalhada das funcionalidades chave implementadas, com especial destaque para o cálculo automático de métricas de concordância, e a documentação técnica produzida.

\section{Apresentação da Plataforma}

A solução desenvolvida é uma aplicação web completa que serve como um ambiente integrado para a anotação de chatrooms, especificamente para a tarefa de \textit{chat disentanglement}. A plataforma foi desenhada para servir dois perfis de utilizador distintos: o \textbf{Anotador}, focado na tarefa de anotação, e o \textbf{Administrador}, responsável pela gestão de projetos, utilizadores e pela análise dos dados.

O fluxo de utilização da aplicação segue um percurso lógico:
\begin{enumerate}
    \item \textbf{Autenticação:} O utilizador acede através de uma página de Login.
    \item \textbf{Dashboard:} Após o login, é apresentado um dashboard adaptado ao seu perfil.
    \begin{itemize}
        \item \textbf{Anotador:} Vê uma lista dos projetos a que está atribuído (\texttt{AnnotatorDashboard}). Ao selecionar um projeto, vê as salas de chat disponíveis (\texttt{AnnotatorProjectPage}).
        \item \textbf{Administrador:} Vê uma visão geral de todos os projetos e utilizadores no sistema (\texttt{AdminDashboard}).
    \end{itemize}
    \item \textbf{Anotação:} O anotador seleciona uma sala de chat e entra na interface de anotação (\texttt{AnnotatorChatRoomPage}), onde pode visualizar as mensagens e atribuir-lhes \textit{thread IDs}.
    \item \textbf{Análise (Admin):} O administrador pode aceder a uma página de projeto (\texttt{AdminProjectPage}) para gerir atribuições, importar dados e, crucialmente, visualizar a análise de concordância entre anotadores (\texttt{AdminChatRoomView}).
\end{enumerate}
% TODO: Esta secção deve ser acompanhada por um diagrama de navegação e screenshots das principais páginas, utilizando os recursos disponíveis em `sitemap_screenshots` para ilustrar o percurso do utilizador.

\section{Resultados da Implementação}

Esta secção detalha as funcionalidades mais relevantes que foram implementadas, demonstrando o cumprimento dos requisitos definidos e respondendo ao feedback recebido pelo júri.

\subsection{Cálculo de Inter-Annotator Agreement (IAA)}

Uma das funcionalidades centrais da plataforma é o cálculo automático da métrica de concordância entre anotadores (IAA), um requisito explícito do júri. A plataforma implementa o algoritmo \textbf{"1-to-1 agreement"}, que avalia a concordância estrutural entre os \textit{threads} criados por diferentes anotadores, focando-se no conteúdo das conversas e não nos nomes arbitrários dos \textit{threads}.

O processo de cálculo, disponível na visão do administrador para cada sala de chat, é o seguinte:
\begin{enumerate}
    \item \textbf{Agregação de Anotações:} Para uma dada sala de chat, o sistema primeiro agrupa todas as anotações por cada anotador. O resultado é um conjunto de "documentos de anotação", um para cada utilizador, onde cada documento contém os \textit{threads} que esse utilizador definiu.
    \item \textbf{Cálculo Par a Par (Pairwise):} A concordância é então calculada para cada par de anotadores possível. Por exemplo, numa sala com três anotadores (A, B, C), o sistema calcula o IAA para (A, B), (A, C), e (B, C).
    \item \textbf{Média Global:} O valor final de IAA apresentado para a sala de chat é a média aritmética de todos os scores de concordância calculados entre os pares.
\end{enumerate}

O núcleo do algoritmo (\texttt{\_calculate\_one\_to\_one\_accuracy} em \texttt{crud.py}) resolve um problema de atribuição. Para cada par de anotadores, ele constrói uma matriz de custo onde a dissimilaridade entre um \textit{thread} do Anotador 1 e um \textit{thread} do Anotador 2 é calculada com base no \textbf{Índice de Jaccard}. O índice mede a sobreposição de mensagens entre os dois \textit{threads}:
\[ J(A, B) = \frac{|A \cap B|}{|A \cup B|} \]
A dissimilaridade na matriz é, portanto, \(1 - J(A, B)\). Com esta matriz, o \textbf{Algoritmo Húngaro} (via \texttt{scipy.optimize.linear\_sum\_assignment}) encontra a correspondência ótima "um-para-um" entre os \textit{threads} que maximiza a semelhança total.

O resultado é uma matriz de similaridade, apresentada na UI do administrador, e um score de concordância final.
% TODO: Incluir uma figura do ecrã de Análise de IAA que mostra a matriz e o score final, explicando como um administrador a interpretaria.

\subsection{Gestão de Projetos e Utilizadores}

A plataforma fornece uma interface de administração robusta que permite a gestão completa do ciclo de vida de um projeto de anotação. As funcionalidades implementadas, acessíveis apenas ao administrador, incluem:
\begin{itemize}
    \item \textbf{Gestão de Projetos:} Criação, listagem e remoção de projetos.
    \item \textbf{Gestão de Utilizadores:} Criação, listagem e remoção de utilizadores.
    \item \textbf{Atribuição a Projetos:} Atribuição granular de utilizadores a projetos específicos, o que garante o isolamento dos dados e a correta distribuição de tarefas.
\end{itemize}

\subsection{Importação e Exportação de Dados}

Para facilitar a integração com outros fluxos de trabalho, foram desenvolvidas funcionalidades de importação e exportação de dados:
\begin{itemize}
    \item \textbf{Importação de Mensagens:} Os administradores podem iniciar um projeto importando uma chatroom completo a partir de um ficheiro CSV.
    \item \textbf{Importação de Anotações:} O sistema suporta a importação em lote de anotações a partir de um ficheiro JSON, que pode conter anotações de múltiplos utilizadores.
    \item \textbf{Exportação de Projetos:} Todos os dados de uma sala de chat (mensagens e a totalidade das anotações) podem ser exportados para um único ficheiro JSON.
\end{itemize}

\section{Documentação Técnica da API}

Uma vantagem inerente à escolha da framework FastAPI é a geração automática de uma especificação da API que segue o standard \textbf{OpenAPI 3.0}. Este processo resulta num ficheiro \texttt{openapi.json} que descreve todos os endpoints da aplicação, os seus parâmetros e os formatos de dados esperados.

O principal objetivo prático deste ficheiro no contexto do projeto foi facilitar os testes e a validação do backend. Ao importar esta especificação para ferramentas de desenvolvimento como o Postman, foi possível testar cada endpoint de forma sistemática e eficiente durante o ciclo de desenvolvimento, garantindo o seu correto funcionamento.

% TODO: Inserir uma pequena tabela ou imagem que resuma os principais grupos de endpoints da API: Auth, Projects, Annotations e Admin, para dar uma visão geral da sua estrutura.

    \chapter{Conclusão}
\label{cha:conclusao}

\section{Conclusão}

O presente Trabalho Final de Curso teve como objetivo central o desenvolvimento de uma ferramenta de software especializada para a tarefa de anotação de \textit{chat disentanglement}. O problema de base identificado foi a ausência de plataformas dedicadas que não só facilitassem o processo de anotação manual, mas que também integrassem mecanismos de análise de qualidade e concordância, forçando frequentemente os investigadores a recorrer a ferramentas genéricas e a processos de cálculo de métricas desligados da tarefa principal.

Em resposta a este desafio, foi concebida, desenvolvida e implementada uma aplicação web completa. A solução, construída sobre uma arquitetura moderna cliente-servidor (React e FastAPI), materializa-se numa plataforma funcional que permite a gestão de projetos de anotação, a atribuição de tarefas a múltiplos anotadores e, mais importante, oferece uma interface otimizada para a tarefa de \textit{chat disentanglement}.

\subsection{Pipeline de Anotação Implementado}

A ferramenta implementa uma arquitetura que suporta o ciclo de vida da anotação através de uma API REST e interfaces especializadas. O sistema integra as seguintes funcionalidades:

\textbf{Ingestão e Processamento de Dados:} A aplicação oferece múltiplas vias para importação de dados de conversas. Através de endpoints REST específicos (\texttt{/api/projects/\{id\}/import-messages}), a ferramenta processa ficheiros CSV contendo mensagens de chat. O sistema realiza validação, estruturação e persistência automática na base de dados SQLite. Esta funcionalidade suporta importação de grandes volumes de dados através de \textit{scripting} ou integração programática. Isto permite que equipas de investigação automatizem a ingestão de \textit{datasets} extensos sem intervenção manual.

\textbf{Interface de Anotação Especializada:} O frontend React implementa uma interface otimizada especificamente para \textit{chat disentanglement}. As funcionalidades incluem visualização cronológica de mensagens, atribuição interativa de \textit{thread IDs}, e gestão de estado em tempo real. A interface comunica com a API através de endpoints dedicados (\texttt{/api/annotations}) que garantem persistência imediata das anotações e sincronização entre múltiplos anotadores.

\textbf{Sistema de Análise Automática:} A ferramenta integra capacidades de análise quantitativa através da implementação do algoritmo "1-to-1 agreement" \cite{elsner2008you}. Este sistema processa automaticamente as anotações de múltiplos utilizadores, calculando métricas de concordância através de técnicas de otimização (Algoritmo Húngaro) e apresentando os resultados através de matrizes de similaridade e scores agregados. O cálculo é acionado dinamicamente através da API (\texttt{/api/admin/chatrooms/\{id\}/iaa}) e apresentado em tempo real na interface administrativa.

\textbf{Exportação e Integração:} A plataforma oferece capacidades de exportação estruturada através de endpoints REST que serializam dados de projetos completos em formato JSON. Esta funcionalidade permite integração com ferramentas de análise externa, facilitando a continuidade do trabalho de investigação e a interoperabilidade com outros sistemas de processamento de linguagem natural.

O principal resultado deste trabalho é uma ferramenta que cumpre os seus requisitos fundamentais. Destaca-se a implementação do cálculo automático do Inter-Annotator Agreement (IAA) através do algoritmo "1-to-1 agreement" \cite{elsner2008you}, que utiliza o Índice de Jaccard e o Algoritmo Húngaro para fornecer uma medida robusta da concordância estrutural entre anotadores. Esta funcionalidade, que foi uma indicação explícita dos coordenadores do TFC, transforma a plataforma de uma simples ferramenta de anotação num ambiente de análise, permitindo aos gestores de projeto aferir a qualidade e a consistência das anotações diretamente no sistema.

No entanto, é com rigor académico que se reconhecem as limitações do trabalho realizado. A principal limitação reside na ausência de uma fase de validação formal com utilizadores finais. Embora a ferramenta seja funcional e tecnicamente robusta, não foi conduzido um estudo empírico para avaliar o seu impacto real na qualidade das anotações ou na experiência do anotador. Adicionalmente, alguns requisitos não-funcionais, como a implementação de backups automáticos e testes de carga formais, foram considerados fora do âmbito da fase de desenvolvimento atual.

\section{Trabalhos Futuros}

As limitações identificadas abrem um caminho claro para trabalhos futuros, que poderiam elevar significativamente o impacto e a maturidade do projeto:

\begin{itemize}
    \item \textbf{Estudo de Validação com Utilizadores:} O passo mais crítico seria a realização de um estudo formal. Este estudo envolveria um grupo de anotadores que realizaria a mesma tarefa de anotação utilizando a nossa ferramenta e um método de base (e.g., folha de cálculo), analisando métricas objetivas (tempo, scores de IAA) e subjetivas (inquéritos de satisfação e usabilidade).
    \item \textbf{Expansão de Métricas de Análise:} A plataforma poderia ser enriquecida com o cálculo de outras métricas de concordância, como o Krippendorff's Alpha, que é mais flexível em cenários com múltiplos anotadores e dados em falta.
    \item \textbf{Generalização da Ferramenta:} A arquitetura foi pensada de forma modular. Um próximo passo seria abstrair o processo de anotação para que a plataforma pudesse ser configurada para outros tipos de tarefas (e.g., anotação de entidades, análise de sentimento).
    \item \textbf{Melhorias de Infraestrutura e Performance:} Implementar as funcionalidades de backup e realizar testes de carga para otimizar o desempenho da API e da base de dados para um número elevado de utilizadores concorrentes, incluindo a migração para um driver de base de dados assíncrono.
\end{itemize}

Em suma, este projeto entregou com sucesso uma solução aplicacional concreta para um problema real no domínio do processamento de linguagem natural, estabelecendo uma base sólida sobre a qual futuras investigações e desenvolvimentos podem ser construídos.

    
    % Appendices
    \appendix
    %\input{appendices/(recomendacoes-formatacao)}
    %\input{appendices/(documentacao-tecnica)}
    %\input{appendices/(prototipos-detalhados)}
    
    \clearpage
    \addcontentsline{toc}{chapter}{Referências Bibliográficas}
    \printbibliography[title={Referências Bibliográficas}]
    \printglossary[title=Glossário,toctitle=Glossário]
\end{document}
